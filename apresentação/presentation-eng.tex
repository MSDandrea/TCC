                %!TEX program = xelatex
\documentclass[9pt, compress]{beamer}
\usetheme[sectionpage=progressbar]{metropolis}

\usepackage{subfigure} 
\usepackage{booktabs}  
\usepackage[scale=2]{ccicons}
\usepackage[T1]{fontenc}
\usepackage[utf8]{inputenc}
\usepackage{lmodern}
\usepackage{amsthm}
\usepackage{diagbox} %tabelas com barras
\usepackage{booktabs}
\usepackage{graphicx}			% Inclusão de gráficos
\usepackage[english]{babel}
\usepackage{svg}

\newtheorem{teorema}{Theorem}
\newtheorem{corolario}{Corolary}
\definecolor{grena}{HTML}{971528}
\definecolor{nice}{HTML}{38b446}
\definecolor{warn}{HTML}{e2cb49}
\renewcommand{\P}{\textcolor{nice}{\textit{P}}}
\newcommand{\NPc}{\textcolor{grena}{\textit{NPc}}}
\newcommand{\?}{\textcolor{warn}{\textit{?}}}

\graphicspath{{../figuras/}}

\author{\textbf{Matheus S. D'Andrea Alves}, \textbf{Uéverton dos Santos Souza} } 
\title{The colourability problem on $(r,\ell)$-graphs.}
\subtitle{And a few parametrized solutions}
%\logo{}
%\institute{\textbf{Universidade Federal Fluminense}}
\date{August 2018}
%\subject{}
%\setbeamercovered{transparent}
%\setbeamertemplate{navigation symbols}{}
\begin{document}
    \maketitle
    \begin{frame}{Topics}
    \centering
        \tableofcontents
    \end{frame}
    \section{The problem}
    \begin{frame}{Basic concepts}
      \begin{columns}
        \begin{column}{0.5\textwidth}
          \textbf{$(r,\ell)$-Graph}
          
          A graph which the vertex set is partitionable into $r$ independent sets and $\ell$ cliques.
        \end{column}
        \begin{column}{0.5\textwidth}
          \textbf{Proper vertex coloring}
          
          A proper vertex coloring of a graph $G$ is the assignment of one element from a set $C$ of $q$ to each vertex in $V(G)$, such that no two adjacent vertices share a common color. A coloring using at most $q$ color is called a (proper) $q$-coloring.
        \end{column}
      \end{columns}
    \end{frame}
    \begin{frame}[standout]
      Our question:
      
      When does this problem become NP-Complete? 
      
      Why?
    \end{frame}
    \section{Our approach}
    \begin{frame}{The strategy}
      Build a dichotomy of the problem, based on the growth of both $r$ and $\ell$.
      
      Use the results to find patterns that exposes the hardness of the problem.
    \end{frame}
    \begin{frame}{First step}
      It is easy to see that:
      \begin{itemize}
        \item A \emph{null} graph (i.e. a graph with 0 vertices) is 0-colorable.
        \item 
        \item 
        \item 
         \item                                                                                                                
      \end{itemize}
    \end{frame}
    \begin{frame}{First step}
      It is easy to see that:
      \begin{itemize}
        \item A \emph{null} graph (i.e. a graph with no vertices) is 0-colorable.
        \item A empty graph (i.e. a graph with no edges) is 1-colorable.
        \item 
        \item 
         \item                                                                                                                
      \end{itemize}
    \end{frame}
    \begin{frame}{First step}
     It is easy to see that:
      \begin{itemize}
        \item A \emph{null} graph (i.e. a graph with no vertices) is 0-colorable.
        \item A empty graph (i.e. a graph with no edges) is 1-colorable.
        \item A bipartite graph (i.e. a graph partitionated into two independent sets) is 2-colorable.
        \item 
         \item                                                                                                                
      \end{itemize}
    \end{frame}
    \begin{frame}{First step}
      It is easy to see that:
      \begin{itemize}
        \item A \emph{null} graph (i.e. a graph with no vertices) is 0-colorable.
        \item A empty graph (i.e. a graph with no edges) is 1-colorable.
        \item A bipartite graph (i.e. a graph partitionated into two independent sets) is 2-colorable.
        \item A complete graph (i.e. a graph in which every pair of distinct vertices is connected ) is $n$-colorable where $n$ is the size of $V(G)$.
        \item                                                                                                               
      \end{itemize}
    \end{frame}
    \begin{frame}{First step}
      It is easy to see that:
      \begin{itemize}
        \item A \emph{null} graph (i.e. a graph with no vertices) is 0-colorable.
        \item A empty graph (i.e. a graph with no edges) is 1-colorable.
        \item A bipartite graph (i.e. a graph partitionated into two independent sets) is 2-colorable.
        \item A complete graph (i.e. a graph in which every pair of distinct vertices is connected ) is $n$-colorable where $n$ is the size of $V(G)$.
        \item A split graph (i.e. a graph partitionated into a independent set and a clique) is $k$-colorable where $k$ is the size of the maximal clique.
      \end{itemize}
    \end{frame}
    \begin{frame}{Partial dichotomy}
        \begin{table}[htb!]
          \center
          \begin{tabular}{l|*{7}c}
            \toprule
            \backslashbox{$r$}{$l$} & 0 & 1 & 2 & 3 & 4 & \ldots & n\\
            \midrule
            0 & \P & \P & \? & \? & \? & \ldots & \?\\
            1 & \P & \P & \? & \? & \? & \ldots & \?\\
            2 & \P & \? & \? & \? & \? & \ldots & \?\\
            3 & \? & \? & \? & \? & \? & \ldots & \?\\
            4 & \? & \? & \? & \? & \? & \ldots & \?\\
            $\vdots$ & $\vdots$ & $\vdots$ & $\vdots$ & $\vdots$ & $\vdots$ & $\ddots$ & \?\\
            n & \? & \? & \? & \? & \? & \ldots & \?\\
            \bottomrule
          \end{tabular}%
          \caption{Partial dichotomy on the Colourability problem in $(r,\ell)$-graphs}
          \label{tab:tabela_part2dictrl}%
        \end{table}%
    \end{frame}
    \begin{frame}{Further}
      \begin{teorema}
        Colourability in $(0,2)$-graphs can be solved in polynomial time.
     \end{teorema}
     \begin{proof}
      A co-bipartite graph is a graph partitionated into two cliques such that every vertex belongs to a clique. It is widely known that any co-bipartite graph is a perfect graph, therefore, your chromatic number is the size of its maximal clique. 
      It has been shown that, to find the maximal clique in a co-bipartite graph is equivalent  to find a vertex cover in its complement, and therefore solvable in polynomial time.
     \end{proof}
    \end{frame}
    \begin{frame}{Further}
      \begin{teorema}
        Colourability in $(3,0)$-graphs can be solved in polynomial time.
     \end{teorema}
     \begin{proof}
      As we know that the given graph is a $(3,0)$-graph, we know that it is $3-colorable$. 
      
      To find out if it is $2$-colorable (i.e bipartite) or $1$-colorable (i.e a empty graph) is solvable in polynomial time. Therefore determining the chromatic number of such graph can be solved in polynomial time.
     \end{proof}
    \end{frame}
    \begin{frame}{Partial dichotomy}
        \begin{table}[htb!]
          \center
          \begin{tabular}{l|*{7}c}
            \toprule
            \backslashbox{$r$}{$l$} & 0 & 1 & 2 & 3 & 4 & \ldots & n\\
            \midrule
            0 & \P & \P & \P & \? & \? & \ldots & \?\\
            1 & \P & \P & \? & \? & \? & \ldots & \?\\
            2 & \P & \? & \? & \? & \? & \ldots & \?\\
            3 & \P & \? & \? & \? & \? & \ldots & \?\\
            4 & \? & \? & \? & \? & \? & \ldots & \?\\
            $\vdots$ & $\vdots$ & $\vdots$ & $\vdots$ & $\vdots$ & $\vdots$ & $\ddots$ & \?\\
            n & \? & \? & \? & \? & \? & \ldots & \?\\
            \bottomrule
          \end{tabular}%
          \caption{Partial dichotomy on the Colourability problem in $(r,\ell)$-graphs}
          \label{tab:tabela_part2dictrl}%
        \end{table}%
    \end{frame}
    \begin{frame}{Further}
      \begin{teorema}
        Colourability in $(4,0)$-graphs is NP-Complete.
     \end{teorema}
     \begin{proof}
      As we know that the given graph is a $(3,0)$-graph, we know that it is $3-colorable$. 
      We need to find out if it can be colored with fewer colors; 
      Note that 3-coloring of planar graphs is NP-Complete, planar graphs are a sub-class of $(4,0)$-graphs, leading to the NP-Completeness of Colourability in $(4,0)$-graphs.
     \end{proof}
    \end{frame}
    \begin{frame}{Partial dichotomy}
        \begin{table}[htb!]
          \center
          \begin{tabular}{l|*{7}c}
            \toprule
            \backslashbox{$r$}{$l$} & 0 & 1 & 2 & 3 & 4 & \ldots & n\\
            \midrule
            0 & \P & \P & \P & \? & \? & \ldots & \?\\
            1 & \P & \P & \? & \? & \? & \ldots & \?\\
            2 & \P & \? & \? & \? & \? & \ldots & \?\\
            3 & \P & \? & \? & \? & \? & \ldots & \?\\
            4 & \NPc & \? & \? & \? & \? & \ldots & \?\\
            $\vdots$ & $\vdots$ & $\vdots$ & $\vdots$ & $\vdots$ & $\vdots$ & $\ddots$ & \?\\
            n & \? & \? & \? & \? & \? & \ldots & \?\\
            \bottomrule
          \end{tabular}%
          \caption{Partial dichotomy on the Colourability problem in $(r,\ell)$-graphs}
          \label{tab:tabela_part2dictrl}%
        \end{table}%
    \end{frame}

    \begin{frame}{Partial dichotomy}
        \begin{table}[htb!]
          \center
          \begin{tabular}{l|*{7}c}
            \toprule
            \backslashbox{$r$}{$l$} & 0 & 1 & 2 & 3 & 4 & \ldots & n\\
            \midrule
            0 & \P & \P & \P & \? & \? & \ldots & \?\\
            1 & \P & \P & \? & \? & \? & \ldots & \?\\
            2 & \P & \? & \? & \? & \? & \ldots & \?\\
            3 & \P & \? & \? & \? & \? & \ldots & \?\\
            4 & \NPc & \NPc & \NPc & \NPc & \NPc & \ldots & \NPc\\
            $\vdots$ & $\vdots$ & $\vdots$ & $\vdots$ & $\vdots$ & $\vdots$ & $\ddots$ & \NPc\\
            n & \NPc & \NPc & \NPc & \NPc & \NPc & \ldots & \NPc\\
            \bottomrule
          \end{tabular}%
          \caption{Partial dichotomy on the Colourability problem in $(r,\ell)$-graphs}
          \label{tab:tabela_part2dictrl}%
        \end{table}%
    \end{frame}
    \begin{frame}[standout]
      How to proceed?
    \end{frame}
    \section{The relationship between coloring and list-coloring in $(r,\ell)$-graphs}
    \begin{frame}{The relationship between coloring and list-coloring in $(r,\ell)$-graphs}
        \begin{teorema}
          List-coloring in $(r,\ell)$-graphs is equivalent to coloring in $(r,\ell+1)$-graphs. 
        \end{teorema}
    \end{frame}
    \begin{frame}{The relationship between coloring and list-coloring in $(r,\ell)$-graphs}
        \textbf{Proof.}
        
        The proof consist in showing that solving list coloring in a $(r,\ell)$-graph $G$, implies in solving coloring for a  $(r,\ell+1)$-graph $H_G$.
        
        In order to proove it, we shall demonstrate:
          \begin{itemize}
        \item If a graph G$(r,\ell)$ has a list coloring then $H_G$ is $k$-colorable where $k$ is the size of the set $C$ of colors (1).
		\item If $H_G$ is $k$-colorable then $G$ has a list coloring (2).
      \end{itemize}
    \end{frame}
    \begin{frame}{The relationship between coloring and list-coloring in $(r,\ell)$-graphs}
      \textbf{(1):}
      
      Let $G$ be a $(r,\ell)$-Graph such that each vertex $v \in V(G)$ has an assigned list of color from $C = \{c_1,c_2,c_3,...,c_k \}$. 
      \begin{center}
        \begin{figure}
        \includesvg[scale=0.4]{presentation-G.svg}
      \end{figure}
      \end{center}
    \end{frame}
    \begin{frame}
      Being $G$ a instance of the list coloring problem, build a clique $K$, where each vertex $u \in V(K)$ represents a color of $C$.
      \begin{center}
      \begin{figure}
        \includesvg[scale=0.4]{presentation-K.svg}
      \end{figure}  
      \end{center}
      Note that $K$ has exactly $k$ vertices, therefore, we can color $K$ with $k$ colors, we can assume that $u_i \in K$ will be colored with $c_i$.
    \end{frame}
    \begin{frame}  
      Let $H_G$ $= G \cup K$, where for each vertex $u_i \in V(K)$ and $v_j \in V(G)$ create an edge $(u_i,v_j) $ in $H_G$ if and only if $c_i$ does not belong to the $v_j$ list.
      \begin{center}
        \begin{figure}
        \includesvg[scale=0.4]{presentation-H.svg}
      \end{figure}
      \end{center}   
      Observe that if we can color $G$, coloring $u_i$ with $c_i$ in $K$ does not conflict with the coloring of $G$, and lead to a coloring of $H_G$. 
    \end{frame}
    \begin{frame}{The relationship between coloring and list-coloring in $(r,\ell)$-graphs}
      \textbf{(2):}
      
      Is given that $H_G$ is a $(r,\ell+1)$-Graph and that it is $k$-colorable.
      
      Note that the removal of $K$ from $H_G$ (becoming $G$) does not affect the coloring, also, to the remaining vertex s $v \in V(G)$ we use its non-neighbourhood to build its list.
      Therefore the coloring of those vertex in $H_G$ is stil valid in $G$.
      
      We have shown that, if $H_G$ is $k$-colorable, $G$ is list-colorable.
      $\qed$
    \end{frame}
    \begin{frame}{Corolários}
      With this result we can assert that:
      \begin{itemize}
        \item Colourability of $(1,2)$-Graphs is NP-Complete.\newline Derives from the demonstration of the NP-Completeness of list coloring in Split graphs presented by Jensen et al. in \textit{"Generalized coloring for tree-like graphs"}.
        \item Colourability of $(2,1)$-Graphs is NP-Complete.\newline Derives from the demonstration of the NP-Completeness of list coloring in bipartite graphs presented by Fellows et al. in \textit{"List Coloring and 3-Precoloring Extension are W[1]-hard parameterized by treewidth"}.
        \item Colourability of $(0,3)$-Graphs is NP-Complete.
        \newline Derives from the demonstration of the NP-Completeness of list coloring in $(0,2)$-Graphs presented by Jensen et al. in \textit{"Complexity results for the optimum cost chromatic partition problem"}.
      \end{itemize}
    \end{frame}
    \begin{frame}{Classic complexity of the Colourability of $(r,\ell)$-graphs}
        The findings fill the table.
        
        \begin{table}[htb!]
          \center
          \begin{tabular}{l|*{7}c}
            \toprule
            \backslashbox{$r$}{$l$} & 0 & 1 & 2 & 3 & 4 & \ldots & n\\
            \midrule
            0 & \P & \P & \P & \NPc & \NPc & \ldots & \NPc\\
            1 & \P & \P & \NPc & \NPc & \NPc & \ldots & \NPc\\
            2 & \P & \NPc & \NPc & \NPc & \NPc & \ldots & \NPc\\
            3 & \P & \NPc & \NPc & \NPc & \NPc & \ldots & \NPc\\
            4 & \NPc & \NPc & \NPc & \NPc & \NPc & \ldots & \NPc\\
            $\vdots$ & $\vdots$ & $\vdots$ & $\vdots$ & $\vdots$ & $\vdots$ & $\ddots$ & \NPc\\
            n & \NPc & \NPc & \NPc & \NPc & \NPc & \ldots & \NPc\\
            \bottomrule
          \end{tabular}%
          \caption{Dichotomy on the Colourability problem in $(r,\ell)$-graphs}
          \label{tab:tabela_dictrl}%
        \end{table}%
    \end{frame}
    \begin{frame}[standout]
      What to look for in $(r,\ell)$-Graphs?
      
    \end{frame}
    \begin{frame}{The approach}
          The problem is NP-Complete.
          
          Can we find a FPT algorithm (i.e. use a parameter $k$ of the Graph such that the problem can be solved in  $\mathcal{O}(f(k)n^c)$)?
          
          Lets start with the partitions size of a $(2,1)$-graph as parameters.
    \end{frame}
    \section{Parameterization by the size of the independet sets}
    
    \begin{frame}{Parameterization}
      \large{Parameterization of $(2,1)$-Graphs by the size of the smallest independent set}
      \normalsize\newline\newline
            
      We know that we can see this problem as list-coloring of a bipartite graph.
      
      Fellows demonstrated that list-coloring is W[1]-hard on bipartites when parameterized by the size of the smallest independent set.\cite{fellows07}
      
      Therefore coloring a $(2,1)$-Graph is W[1]-hard when parameterized by the size of the smallest independent set.
    \end{frame}
    \begin{frame}{Parameterization}
    \large{Parameterization of $(2,1)$-Graphs by the size of the largest independent set}
      \normalsize\newline\newline
            
      We know that we can see this problem as list-coloring of a bipartite graph.
      
      To list-color a bipartite, if a vertex has a list larger than its neighbourhood, it will always have an available color, we can then remove this vertex without affecting the graph coloring.
      
      Note that after the removal of those vertices both the number of vertices and size of lists are bounded by $k$,  therefore applying a brute force algorithm give us a FPT algorithm to solve the problem.
     \end{frame}
     
    \section{Parameterization by the size of the clique}
     \begin{frame}
       We want to understand how the problem behave when the graph has a small clique.
       
       Colourability of $(2,1)$-Graph when the clique is small, is equivalent to list-coloring a bipartite graph with a small set of colors.
       
       The following problem help us understand the behavior of list-coloring this bipartite.
     \end{frame}
    
     \begin{frame}{$k$-PreColoring extension}
      \large{$k$-PreColoring extension}
      \normalsize\newline\newline
      
	      \textbf{Entrada:}  A graph $G$ such that some of its vertex are already colored with one of $k$ possible colors.
	      
	      \textbf{Pergunta:}  Can we extend this coloring obtaining a proper coloring to this graph?
	      
	      This problem is shown to be NP-Complete even when $k=3$(3-PreCloring extension).\cite{kratochvil94}
     \end{frame}
     
     \begin{frame}{3-list coloring on bipartite graphs is NP-Complete.}
    \large{3-list coloring on bipartite graphs is NP-Complete.}
      \normalsize\newline\newline
            
      Having a graph $G$ and a set of colors $C$ from a instance of \emph{3-PreColoring extension}.
      
      We build a graph $G'$ using every pre-colored $v \in V(G)$ giving it a list in $G'$ containing its color in $G$, the remaining vertices of $G$ are mapped in $G'$ using lists containing all three colors, we keep the neighbourhood of each vertex.
      
      If we can color $G$ then we can list-color $G'$ , since we only need to asign to the vertex in $G'$ the same colors chosen in $G$. Analogously, if we can list-color $G'$ then we can color $G$.
      
      
     \end{frame}
     \begin{frame}{Parameterization.}
    \large{Parameterization of a $(2,1)$-graph by the clique size.}
      \normalsize\newline\newline
            
      To demonstrate the parameterized intractability, we will show that the problem remains hard even when the clique is a triangle.
      
      When the parameterizing by clique size, this parameterization can be seen as parameterizing list-coloring by the size of the color set, that we have already demonstrated to be impractical even for three colors.
      
     Therefore this problem remains NP-Complete even when when the clique is a triangle, and therefore not usefull to be used as a parameter.
      
     \end{frame}
     \begin{frame}[standout]
       The results still aren't satisfactory.
       
       What else can we observe from this class?
     \end{frame}
     
     \begin{frame}{Observations}
       Using a $(2,1)$-Graph where the maximal clique is a triangle, our problem become 3-list-coloring in bipartite..
       
       3-coloring of bipartite is Polynomial.
       
       2-coloração of bipartite is Polynomial. 
       
       3-list-coloring of bipartite is NP-Complete.
       
       What happens when the amount of vertices with lists of size 1, 2 e 3 varies? 
     \end{frame}
     \begin{frame}{The clique neighbourhood}
       Note that if a vertex is not in the neighbourhood of any of the vertex from the clique, after our reduction it has list of size three. Every vertex neighbor of the clique, after the reduction, is a vertex with list of size $3-x$ where $x$ is the number of neighbours belonging to the clique.
       
       No vertex can have lists of size 0.
     \end{frame}
     \section{Parameterizing by the size of the clique neighbourhood}
     \begin{frame}{Vertices with unitary lists}
       
    Note the following gadget.
   \begin{figure}[H]
      \begin{subfigure}
        \centering
		    \includesvg[scale=0.4]{gadget-1.svg}
      \end{subfigure}
      \begin{subfigure}
        \centering
		    \includesvg[scale=0.4]{gadget-2.svg}
      \end{subfigure}
      \begin{subfigure}
        \centering
		    \includesvg[scale=0.4]{gadget-3.svg}
      \end{subfigure}
      \caption{Gadget of vertices with unitary lists forcing a unitary list on a vertex with three colors}
      \label{fig:gadget}
  \end{figure}
    \end{frame}
    \begin{frame}{Vértices com listas de tamanho um}
      Using this gadget is possible to transform any instance of 3-PreColoring Extension on bipartite into list coloring of bipartite, since we just need to use the vertices from $G$ that are pre-colored to create a correspondent vertex in $G'$ with lists of size three, by connecting them to vertex of unitary lists in order to emulate its color in $G$, the remaining vertices of $G$ are mapped to vertices in $G'$ with lists of size three and the neighbourhood of each vertex persisted.
      
    \end{frame}
     \begin{frame}{Vertices with unitary lists}
       \begin{teorema}
         Three vertices with unitary lists are enough for list-coloring in bipartite to be NP-Complete.
       \end{teorema}
       \begin{proof}
        Using the gadget, is possible to obtain enough vertices on each partition to allow a reduction from any instance of 3-Precoloring extension on biartite to list coloring in bipartite.
       \end{proof}
     \end{frame}
     \begin{frame}
       \begin{figure}[H]
        \centering
        \fontsize{4}{10}
        \includesvg[scale=0.5]{g'.svg}
      \end{figure}
     \end{frame}
     \begin{frame}
       \begin{figure}[H]
        \centering
        \fontsize{4}{10}
        \includesvg[scale=0.3]{g'1.svg}
      \end{figure}
     \end{frame}
     \begin{frame}
       \begin{figure}[H]
        \centering
        \fontsize{4}{10}
        \includesvg[scale=0.3]{g'2.svg}
      \end{figure}
     \end{frame}
     \begin{frame}
       \begin{figure}[H]
        \centering
        \fontsize{4}{10}
        \includesvg[scale=0.3]{g'4.svg}
      \end{figure}
     \end{frame}
     \begin{frame}{Vértices com listas de tamanho dois}
       \begin{teorema}
         Six vertices with lists of size two are enough to list-coloring in bipartite to be NP-Complete.
       \end{teorema}
       \textbf{Proof.}
       
 
       In order to demonstrate this, we need to find a configuration of the bipartite where the problem remains NP-Complete.
       
       Lets look into a specific instance of the neighbourhood of those six vertices. 
       
     \end{frame}
     
     \begin{frame}{Every one of the six vertices has neighbourhood of size two and no color repeats in all lists}
      Those restrictions lead us to the $\Gamma$ structure its two possible colorings can be:

      \begin{figure}[H]
        \centering
        \fontsize{4}{10}
        \includesvg[scale=0.3]{2-edge-b.svg}
        \caption{$\Gamma$ and its colorings.}
        \label{fig:2-edge-b}
      \end{figure}
      Therefore to reduce the 3-PreColoring extension $G$ to our problem $G'$ te we just need to for every pre-colored vertex  $v_{c_j} \in V(G) | c_j \in C$ create a vertex $u \in V(G')$ with list of size three and connect it to the vertices of $\Gamma$ in the following way:
      
     \end{frame}
     \begin{frame}
       \begin{columns}
        \begin{column}{0.5\textwidth}
          Choose a vertex $v \in A$ tha is pre-colored with color $c_j$, note that $\Gamma$ have two ways of forcing the color $c_j$ in $u$ the gadget showed in figure \ref{fig:gadget}, choose one of this ways. Therefore $\forall v_{c_j} | j \neq i$ we connect it to $\Gamma$ in such way that it respects the connections that already induces a color.the remaining vertices of $G$ are mapped to vertices in $G'$ with lists of size three and the neighbourhood of each vertex persisted.
        \end{column}
        \begin{column}{0.5\textwidth}
        \end{column}
      \end{columns}
     \end{frame}
     \begin{frame}{Parameterized by the size of the non-neighbourhood of the clique}
       
        \begin{teorema}
        List-Coloring of bipartite wiwth lists of size one to three is FPT when parameterized by the amount of vertices with lists of size three.
        \end{teorema}
        \begin{proof}
         Since we have $k$ vertices with 3 choices of colors each, is possible to obtain a bounded search tree algorithm  in a tree of size $3^k$ that lead to a 2-list coloring of bipartite, obtaining a algorithm $\mathcal{O}(3^kn^{c})$
\end{proof}
     \end{frame}
     
     \section{Conclusion}
     \begin{frame}{Classic results}
       \begin{table}[htb!]
          \center
          \begin{tabular}{l|*{7}c}
            \toprule
            \backslashbox{$r$}{$l$} & 0 & 1 & 2 & 3 & 4 & \ldots & n\\
            \midrule
            0 & \P & \P & \P & \NPc & \NPc & \ldots & \NPc\\
            1 & \P & \P & \NPc & \NPc & \NPc & \ldots & \NPc\\
            2 & \P & \NPc & \NPc & \NPc & \NPc & \ldots & \NPc\\
            3 & \P & \NPc & \NPc & \NPc & \NPc & \ldots & \NPc\\
            4 & \NPc & \NPc & \NPc & \NPc & \NPc & \ldots & \NPc\\
            $\vdots$ & $\vdots$ & $\vdots$ & $\vdots$ & $\vdots$ & $\vdots$ & $\ddots$ & \NPc\\
            n & \NPc & \NPc & \NPc & \NPc & \NPc & \ldots & \NPc\\
            \bottomrule
          \end{tabular}%
          \caption{Dichotomy of the Colourability problem in $(r,\ell)$-Graphs}
          \label{tab:tabela_dictrl}%
        \end{table}
     \end{frame}
     \begin{frame}{Colateral results}
     Is widely known that the colourability problem in a graph $G$ can be treated as the clique cover problem in its complement  $G'$\cite{gareyjohnson}, we can then extend the dichothomy changing $r$ by $\ell$ and obtain a dichotomy to the clique cover problem.
\begin{table}[!htb]
	\center
	\begin{tabular}{l|*{7}c}
		\toprule
		\backslashbox{$r$}{$l$} & 0 & 1 & 2 & 3 & 4 & \ldots & n\\
		\midrule
		0 & \P & \P & \P & \P & \NPc & \ldots & \NPc\\
		1 & \P & \P & \NPc & \NPc & \NPc & \ldots & \NPc\\
		2 & \P & \NPc & \NPc & \NPc & \NPc & \ldots & \NPc\\
		3 & \NPc & \NPc & \NPc & \NPc & \NPc & \ldots & \NPc\\
		4 & \NPc & \NPc & \NPc & \NPc & \NPc & \ldots & \NPc\\
		$\vdots$ & $\vdots$ & $\vdots$ & $\vdots$ & $\vdots$ & $\vdots$ & $\ddots$ & \NPc\\
		n & \NPc & \NPc & \NPc & \NPc & \NPc & \ldots & \NPc\\
		\bottomrule
	\end{tabular}%
	\caption{$\P/\NPc$ dichotomy of clique cover in $(r,\ell)$-graphs}%
\end{table}%
     \end{frame}
     \begin{frame}{Parameterized results}
       \begin{itemize}
         \item 2 FPT algorithm.
         \item One $W[1]$-hardness demonstration.
         \item Some para-NP-completeness demonstration.
       \end{itemize}
     \end{frame}
     \begin{frame}{Future work}
       \begin{itemize}
  \item  Is there a relationship between the parameterization of the colourability problem in $(2,1)$-Graphs and the parameterization of clique cover in $(1,2)$-Graphs?
  \item Which characteristics affects the parameterization of $(r,\ell)$-Graphs? Are the different from $(2,1)$-Graphs?
  \item Does exists a parameter such that is possible to solve this problem with a FPT algorithm to any $r$ and $\ell$?
\end{itemize}
     \end{frame}
     \begin{frame}[standout]
       Thank you!
       
       Questions?
     \end{frame}
     \begin{frame}{References}
       \bibliography{bibliografia}
       \bibliographystyle{plain}
     \end{frame}
\end{document}
