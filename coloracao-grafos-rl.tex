\documentclass[a4paper,oneside,12pt]{book}
%\documentclass[a4paper,oneside]{book}
\pagestyle{myheadings}

\setcounter{secnumdepth}{3}

%%pacote para tabela
\usepackage{booktabs}
\usepackage{graphicx}
\usepackage{adjustbox}
\usepackage[normalem]{ulem}
\useunder{\uline}{\ul}{}
\usepackage{proof,bbm}
\usepackage{stmaryrd}
\usepackage{amsmath}
\usepackage{esvect}

%%pacote de c�digos
\usepackage{listings}
\lstdefinestyle{padrao}{
	basicstyle=\ttfamily\scriptsize,
	columns=fullflexible,
	mathescape=true,
	tabsize=2, linewidth=\textwidth, 
	numbers=left,
	numberstyle=\tiny,
	stepnumber=1,
	numbersep=5pt,
	backgroundcolor=\color{gray!20},
	showspaces=false,
	showstringspaces=false,
	showtabs=false,
	frame=lines,
	tabsize=2,
	captionpos=b,
	floatplacement={tbp},
	breaklines=true,
	breakatwhitespace=false,
	escapeinside={\%*}{*)},
	numberbychapter=false
}


%%%%%%%%%%%%%%%%%%%%%%%%%%%
% Pacotes para acentua��o %
%%%%%%%%%%%%%%%%%%%%%%%%%%%

%Apenas para testar
\usepackage{cmap}
%Apenas para teste

\usepackage[brazilian]{babel}
\usepackage[utf8]{inputenc}
\usepackage[T1]{fontenc}
\usepackage{ae}
\usepackage{courier}


%%%apendice
\usepackage[titletoc]{appendix}

%%%%define uma tabela muito longa
\usepackage{longtable}

%%cor para tabela
\usepackage[table,xcdraw]{xcolor}

%%%%%%%%%%%
%\usepackage[brazilian]{babel}
\usepackage{graphicx}
\usepackage{placeins}
\usepackage{tikz}
\usepackage{amsfonts}
\usepackage{hyperref}
\usetikzlibrary{arrows,shapes,decorations,automata,backgrounds}
\usepackage{amsthm}
\usepackage{hyperref}





\usepackage{lmodern}			% Usa a fonte Latin Modern			
\usepackage[T1]{fontenc}		% Selecao de codigos de fonte.
\usepackage[utf8]{inputenc}		% Codificacao do documento (conversão automática dos acentos)
\usepackage{lastpage}			% Usado pela Ficha catalográfica
\usepackage{indentfirst}		% Indenta o primeiro parágrafo de cada seção.
\usepackage{color}		% Controle das cores
\usepackage{graphicx}			% Inclusão de gráficos
\usepackage{microtype} 			% para melhorias de justificação
\usepackage{diagbox} %tabelas com barras
% ---
		
% ---
% Pacotes adicionais, usados apenas no âmbito do Modelo Canônico do abnteX2
% ---
\usepackage{lipsum}				% para geração de dummy text
% ---

% ---
% Pacotes de citações
% ---
\usepackage[brazilian,hyperpageref]{backref}	 % Paginas com as citações na bibl
\usepackage[alf]{abntex2cite}	% Citações padrão ABNT

% --- 
% Pacotes matemáticos
% ---
\usepackage{amsthm}


% --- 
% CONFIGURAÇÕES DE PACOTES
% --- 
\newtheorem{definition}{Definição}
\newtheorem{teorema}{Teorema}
\newtheorem{corolario}{Corolário}
\graphicspath{{figuras/}}
% ---
\newtheorem{axiom}{Axioma}


\newcommand{\Mod}[1]{\ (\mathrm{mod}\ #1)}
%%%%%%%%%%%
\linespread{1.5} % espaçamento entre linhas
%%% Outros pacotes úteis - Igor - 05/11/2011
%%%%%%%%%%%%%%%%%%%%%%%%%%%%%%%%%%%%%%%%%%
%%% Insira aqui os pacotes necess�rios
%%%%%%%%%%%%%%%%%%%%%%%%%%%%%%%%%%%%%%
\usepackage{indentfirst}
\usepackage{array}

%%%%%%%%%%%%%%%%%%%%%%%%%%%%%%%%%%%%%%%%%%%%%%%%%%%%%%
%              Formata��o da P�gina                  %
%%%%%%%%%%%%%%%%%%%%%%%%%%%%%%%%%%%%%%%%%%%%%%%%%%%%%%

% horizontal
\setlength{\hoffset}{-1in}

\setlength{\oddsidemargin}{3.0cm}

\setlength{\textwidth}{160mm} % (210mm - 30mm - 20mm)

\setlength{\parindent}{1.25cm} % identa��o de cada par�grafo

% vertical
\setlength{\voffset}{-1in}
\addtolength{\voffset}{2.0cm}

\setlength{\topmargin}{0.0cm}

\setlength{\headheight}{5mm}
\setlength{\headsep}{5mm}

\setlength{\textheight}{247mm} % (297mm - 30mm - 20mm)

%\setlength{\footskip}{0mm}

%%%%%%%%%%%%%%%%%%%%%%%%%%%%%%%%%%%%%%%%%%%%%%%%%%%%%%


\begin{document}


%%%%%%%%%%%%%%%%%%%%%%%%%%%%%%%%%%%%%%%%%%%%%%%%%%%%%%
%                  Capa da Monografia                %
%%%%%%%%%%%%%%%%%%%%%%%%%%%%%%%%%%%%%%%%%%%%%%%%%%%%%%

\begin{titlepage}
  \begin{center}
    \Large{\textsc{Universidade Federal Fluminense} \\
           \textsc{Instituto de Computação} \\
           \textsc{Departamento de Ciência da Computação}
          }
    \par\vspace{3.0cm}
    \LARGE{Matheus Souza D'Andrea Alves}
%% Descomentar caso tenha outro aluno
     \par\vspace{3.0cm}
    \bigskip
    \LARGE{COLORAÇÃO DE $GRAFOS(r,l)$}
    \par\vfill
    \Large{Niterói-RJ\\2017}
  \end{center}
\end{titlepage}



%%%%%%%%%%%%%%%%%%%%%%%%%%%%%%%%%%%%%%%%%%%%%%%%%%%%%%
%                Numeracao em romano                 %
%%%%%%%%%%%%%%%%%%%%%%%%%%%%%%%%%%%%%%%%%%%%%%%%%%%%%%

\pagenumbering{roman}
\setcounter{page}{2}


%%%%%%%%%%%%%%%%%%%%%%%%%%%%%%%%%%%%%%%%%%%%%%%%%%%%%%
%                   Folha de Rosto                   %
%%%%%%%%%%%%%%%%%%%%%%%%%%%%%%%%%%%%%%%%%%%%%%%%%%%%%%


\begin{center}
Matheus Souza D'Andrea Alves


\vfill

COLORAÇÃO DE $GRAFOS(r,l)$

\vspace{3.0cm}

\begin{flushright}
\begin{minipage}{0.50\textwidth}

Trabalho submetido ao Curso de \linebreak Bacharelado em Ciência da
Computação
da Universidade Federal Fluminense como
requisito parcial para a obtenção do título de Bacharel em Ciência da
Computação.

\end{minipage}
\end{flushright}

\vspace{3.0cm}

\begin{flushleft}
Orientador: Dr. Uéverton dos Santos Souza 
\end{flushleft}

\vfill

Niterói-RJ\\2017

\end{center}

\newpage


\newpage

%%%%%%%%%%%%%%%%%%%%%%%%%%%%%%%%%%%%%%%%%%%%%%%%%%%%%%%%
%                  Dedicat�ria                                 %
%%%%%%%%%%%%%%%%%%%%%%%%%%%%%%%%%%%%%%%%%%%%%%%%%%%%%%%%

\begin{flushright}
\begin{minipage}{0.5\textwidth}

\vspace{15.0cm}
% espa�o do topo at� o in�cio da dedicat�ria



\textit{"Ignorance more frequently begets confidence than does knowledge:\\
		It is those who know little, and not those who know much,\\ who so positively assert that this or that problem\\ will never be solved by science.\\
		Charles Darwin (The Descent of Man - pg 3)}
\end{minipage}
\end{flushright}

%%%%%%%%%%%%%%%%%%%%%%%%%%%%%%%%%%%%%%%%%%%%%%%%%%%%%%%%
%                 Agradecimentos                             %
%%%%%%%%%%%%%%%%%%%%%%%%%%%%%%%%%%%%%%%%%%%%%%%%%%%%%%%%

\chapter*{Agradecimentos}

\thispagestyle{myheadings}

\noindent

[PLACE HOLDER]



%%%%%%%%%%%%%%%%%%%%%%%%%%%%%%%%%%%%%%%%%%%%%%%%%%%%
%            Resumo na l�ngua vern�cula            %
%%%%%%%%%%%%%%%%%%%%%%%%%%%%%%%%%%%%%%%%%%%%%%%%%%%%

\chapter*{Resumo}
\addcontentsline{toc}{chapter}{Resumo}

\thispagestyle{myheadings}

Um problema clássico na literatura é o problema de coloração própria de um grafo, isto é, encontrar uma q-coloração para um grafo G tal que todo vértice $v \in V(G)$ não possua nenhum vizinho da mesma cor e q seja mínimo. Esse problema é conhecido ser NP-Difícil para grafos gerais. O trabalho a seguir tem como proposta desvendar e catalogar a complexidade clássica e parametrizada de tal problema para a classe de Grafos(r,l), i.e. grafos particionáveis em r conjuntos independentes e l cliques; Identificando as características que tornam o problema difícil e a relação do problema de coloração com outros problemas, quando abordado pela perspectiva parametrizada.

\bigskip
%

\noindent Palavras-chave: \textit{Complexidade parametrizada. Grafos(r,l). Partição de grafos. Coloração de Grafos}

%%%%%%%%%%%%%%%%%%%%%%%%%%%%%%%%%%%%%%%%%%%%%%%%%%%%%%
%                      Abstract                      %
%%%%%%%%%%%%%%%%%%%%%%%%%%%%%%%%%%%%%%%%%%%%%%%%%%%%%%

\chapter*{Abstract}
\addcontentsline{toc}{chapter}{Abstract}

\thispagestyle{myheadings}
\nocite{*}

A classical problem in the literature is the problem of proper coloring a graph, i.e. to find a q-coloring for a graph G such that every vertex $ v \in V (G) $ does not have any neighbor of the same color and q is the smallest possible number, a problem known to be NP-Hard for a general graphs. The following work attempts to uncover and catalog the parametrized complexity of such problem for the class of graphs(r, l), i.e. partitionable graphs in r independent sets and l cliques; Identifying the characteristics that make the problem hard and the relation of the stated problem to other problems when approached by the parameterized perspective.

\bigskip
%

\noindent Keywords: \textit{Parametrized Complexity. Graph(r,l). Graph Partitioning. Graph Coloring. }

%%%%%%%%%%%%%%%%%%%%%%%%%%%%%%%%%%%%%%%%%%%%%%%%%%%%%%%%
%                       Sum�rio                        %
%%%%%%%%%%%%%%%%%%%%%%%%%%%%%%%%%%%%%%%%%%%%%%%%%%%%%%%%

\tableofcontents

\thispagestyle{myheadings}


%%%%%%%%%%%%%%%%%%%%%%%%%%%%%%%%%%%%%%%%%%%%%%%%%%%%%%%%
%                  Lista de Ilustra��es                %
%%%%%%%%%%%%%%%%%%%%%%%%%%%%%%%%%%%%%%%%%%%%%%%%%%%%%%%%

\listoffigures
\addcontentsline{toc}{chapter}{Lista de Figuras}

\thispagestyle{myheadings}

%%%%%%%%%%%%%%%%%%%%%%%%%%%%%%%%%%%%%%%%%%%%%%%%%%%%%%%%
%                   Lista de Tabelas                   %
%%%%%%%%%%%%%%%%%%%%%%%%%%%%%%%%%%%%%%%%%%%%%%%%%%%%%%%%

\listoftables
\addcontentsline{toc}{chapter}{Lista de Tabelas}

\thispagestyle{myheadings}


%%%%%%%%%%%%%%%%%%%%%%%%%%%%%%%%%%%%%%%%%%%%%%%%%%%%%%
%                Numeracao em arabico                %
%%%%%%%%%%%%%%%%%%%%%%%%%%%%%%%%%%%%%%%%%%%%%%%%%%%%%%

\pagebreak
\pagenumbering{arabic}

%%%%%%%%%%%%%%%%%%%%%%%%%%%%%%%%%%%%%%%%%%%%%%%%%%%%%%%%
%                       Texto                          %
%%%%%%%%%%%%%%%%%%%%%%%%%%%%%%%%%%%%%%%%%%%%%%%%%%%%%%%%




%%% Separe cada cap�tulo em um arquivo separado
%%% Os arquivos podem ter qualquer nome
\chapter{Introdução} \label{cap:introducao}
\chapter{Preparação da pesquisa}
\include{capitulos/analiseClassicaColoracao}
\chapter{Análise parametrizada para coloração em Grafos(2,1)}
Tendo mostrado a complexidade clássica nos é interessante agora que elucidemos quais caractéristicas dos grafos(r,l) se mostram propícias a abordagem parametrizada, a cardinalidade de suas partições se mostrou uma interesante característica.
Decidimos abordar a classe (2,1), já que a mesma é a classe onde o problema é NP-Completo com o menor número de partições.

Um Grafo(2,1) é um grafo particionado em 2 conjuntos independentes e 1 clique, portanto ele nos entrega 3 naturais candidatos a parametrização, o tamanho da clique $l$, o tamanho do menor conjunto independente $r_1$ e o tamanho do maior conjunto independente $r_2$.

\section{Parametrização pelo tamanho do menor conjunto independente}
Em \ref{Fellows07} Fellows (et. al) mostrou que o problema de lista coloração é $W[1]-difícil$ parametrizado pela treewidth através da transformação do problema da clique multicolorida parametrizada pelo tamanho da clique para tal, nos aproveitaremos dessa transformação para mostrar que:
\begin{teorema}
O problema de coloração em Grafos(2,1) é $W[1]-difícil$ quando parametrizado pelo tamanho do menor conjunto independente.
\end{teorema}
\begin{proof}
Observe a seguinte transformação.

O problema da clique multicolorida pode ser definido como: Dado um Grafo $G$ detentor de uma $k-coloração$ própria, existe uma clique de tamanho $k$ abrangendo todas as cores? Esse problema é conhecidamente $W[1]-difícil$.

Portanto suponha tal $G$ proposto, temos como intenção montar um problema de lista coloração em um grafo $G'$ a partir dele, para tanto seguimos os seguintes passos:
\begin{itemize}
  \item Para cada cor $i$ presente em $G$ cria-se em $G'$ um vértice $v_i$ (os chamaremos de vértices-cor).
  \item Para cada vértice $u$ em $G$ colorido com a cor $i$, adicionamos à lista do vértice-cor $v_i$ em $G'$ uma cor $c_u$ relacionada a esse vértice (as chamaremos de cores-vértice).
  \item Para cada aresta $e(x,y) \notin E(G)$ onde $x,y \in V(G)$ cria-se em $G'$ um vértice $z_e$ adjacente ao vértice-cor $v_i$ onde $i$ representa as cores de $x$ e $y$, a lista coloração de $z_e$ será formada por $c_x$ e $c_y$.
\end{itemize}
É notável que a treewidth de $G'$ é dada por $k$, já que a remoção dos vértices-cor leva a um grafo sem arestas. Assim sendo se $G$ possui uma clique multicolorida podemos facilmente colorir $G'$ da seguinte forma:

Ao vértice-cor $v_i$ atribua a cor-vértice $c_u$ onde $u$ é o vértice colorido com a cor $i$ em $G$. Dessa forma todos os vértices $z_e$ possuem ainda uma cor disponível para sua coloração já que ele representa uma não-aresta em $G$. 

Para a volta observe que uma lista coloração válida em $G'$ implica em uma clique multicolorida em $G$, isso se dá pois dois vértices $x,y$ coloridos com cores diferentes em $G$ não aparecem em uma lista de algum $z_e$ em $G'$ se e somente se existe uma aresta $e(x,y) \in E(G)$, portanto as cores-vértices escolhidas para os vértices $v_i$ são uma respectivamente uma clique formadas por tais $i$ em $G$. Mostramos assim que lista coloração parametrizada por treewidth é $W[1]-difícil$.

Sabemos que coloração em Grafos(2,1) é equivalente a lista coloração em um grafo bipartido, portanto nossa tentativa de parametrizar a coloração de (2,1) pelo tamanho do menor conjunto independente é equivalente a parametrizar lista-coloração em bipartidos pelo tamanho do menor conjunto independente, é de pouca dificuldade ver que a treewidth de um grafo bipartido existe em função do menor independente, mostrando assim que coloração em Grafos(2,1) parametrizada pelo tamanho do menor conjunto independente é $W[1]-difícil$. 

\end{proof}

\section{Parametrização pelo tamanho da clique}
Para a demonstração da complexidade parametrizada utilizando $k=\#l$ nos voltamos novamente para transformação de uma coloração em Grafo(2,1) para Lista coloração em bipartido, dessa forma nosso problema parametrizado original se torna um novo problema, lista coloração de bipartido parametrizado pela paleta de cores, já que cada vértice pertencente a clique se transforma em uma cor na paleta do bipartido. 

Mostraremos porém que essa parametrização não é util já que o problema se transforma em PreColoring Extension com limite de cores, mostrado ser NP-Completo\cite{Kratochvil94} para grafos bipartidos mesmo quando sua paleta é de tamanho 3.

\begin{teorema}
Lista coloração de Bipartido com Listas de tamanho 1 a 3 é NP-Completo
\end{teorema}
\begin{proof}
Suponha um grafo G, onde existem apenas listas de tamanho 1 e 3, é trivial notar que como somos obrigados a colorir os vértices com lista de tamanho 1 com a única cor nela disponível, e todos os outros vértices possuem todas as 3 cores disponíveis então esse problema é na verdade o problema de PreColoring Extension com limite de cores 3.
\end{proof}
\begin{corolario}
Coloração de Grafo(r,l) parametrizado pelo tamanho da clique é Para-NP
\end{corolario}
\begin{proof}
<Não sei como escrever>
\end{proof}

Apesar do tamanho da paleta não ter se mostrado uma escolha adequada, ele levanta novos parametros que são interessantes para o problema de lista coloração em bipartidos, observe pois que, sabemos que Precoloring extension é polinomial se todas as listas tem tamanho 1 ou 2 \cite{HUJTER93}, e NP-Completo se todas tem listas e tamanho 1 e 3 \cite{Kratochvil94}, isso levanta duas formas de se abordar o problema, o que acontece quando o número de vértices com listas de tamanho 1 e 2 varia, e o que acontece quando o número de vertices com listas de tamanho 3 varia.

Mostraremos nas próximas seções como se dão tais comportamentos e como eles se relacionam a coloração de Grafos(r,l)
\section{O problema de lista coloração em bipartidos com paleta de tamanho 3 parametrizado pela quantidade de vértices com listas de tamanho 1 ou 2}
Subdiviremos essa seção abordando os casos em que existem apenas listas de tamanho 1, e onde existem listas de tamanho 2.
\subsection{Apenas listas de tamanho 1}
 <Não lembro como,> sabemos que precisamos de 6 vértices de tamaho 1 para que o problema se 
\section{O problema de lista coloração em bipartidos com paleta de tamanho 3 parametrizado pela quantidade de vértices com listas de tamanho 3}
Mostraremos que o problema é FPT utilizando o seguinte algoritmo de árvore de profundidade limitada.
\chapter{Análise parametrizada de problemas relacionados a coloração}
\chapter{Resultados}
\chapter{Conclusão}
%%%%%%%%%%%%%%%%%%%%%%%%%%%%%%%%%%%%%%%%%%%%%%%%%%%%%%%%%%%%%%%%%%%
%                  Referencias Bibliograficas                   %  
%%%%%%%%%%%%%%%%%%%%%%%%%%%%%%%%%%%%%%%%%%%%%%%%%%%%%%%%%%%%%%%%%%%
\cleardoublepage
\addcontentsline{toc}{chapter}{Referências Bibliográficas}

%\appendix
%\include{apendice1}
%\apendice
%\cleardoublepage
%\addcontentsline{toc}{chapter}{Ap\^endices}
%\begin{appendices}
%\chapter{Teste}
%\end{appendices}
\bibliographystyle{IEEEtran}
\bibliography{references}%{} % arquivos fonte com a bibliografia
\end{document}
