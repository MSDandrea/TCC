\documentclass[a4paper,oneside,12pt]{book}
%\documentclass[a4paper,oneside]{book}
\pagestyle{myheadings}

%%pacote para tabela
\usepackage{booktabs}
\usepackage{graphicx}
\usepackage{adjustbox}
\usepackage[normalem]{ulem}
\useunder{\uline}{\ul}{}
\usepackage{proof,bbm}
\usepackage{stmaryrd}
\usepackage{amsmath}
\usepackage{esvect}

%%pacote de c�digos
\usepackage{listings}
\lstdefinestyle{padrao}{
	basicstyle=\ttfamily\scriptsize,
	columns=fullflexible,
	mathescape=true,
	tabsize=2, linewidth=\textwidth, 
	numbers=left,
	numberstyle=\tiny,
	stepnumber=1,
	numbersep=5pt,
	backgroundcolor=\color{gray!20},
	showspaces=false,
	showstringspaces=false,
	showtabs=false,
	frame=lines,
	tabsize=2,
	captionpos=b,
	floatplacement={tbp},
	breaklines=true,
	breakatwhitespace=false,
	escapeinside={\%*}{*)},
	numberbychapter=false
}


%%%%%%%%%%%%%%%%%%%%%%%%%%%
% Pacotes para acentua��o %
%%%%%%%%%%%%%%%%%%%%%%%%%%%

%Apenas para testar
\usepackage{cmap}
%Apenas para teste

\usepackage[brazilian]{babel}
\usepackage[utf8]{inputenc}
\usepackage[T1]{fontenc}
\usepackage{ae}
\usepackage{courier}


%%%apendice
\usepackage[titletoc]{appendix}

%%%%define uma tabela muito longa
\usepackage{longtable}

%%cor para tabela
\usepackage[table,xcdraw]{xcolor}

%%%%%%%%%%%
%\usepackage[brazilian]{babel}
\usepackage{graphicx}
\usepackage{placeins}
\usepackage{tikz}
\usepackage{amsfonts}
\usepackage{hyperref}
\usetikzlibrary{arrows,shapes,decorations,automata,backgrounds}
\usepackage{amsthm}
\usepackage{hyperref}





\usepackage{lmodern}			% Usa a fonte Latin Modern			
\usepackage[T1]{fontenc}		% Selecao de codigos de fonte.
\usepackage[utf8]{inputenc}		% Codificacao do documento (conversão automática dos acentos)
\usepackage{lastpage}			% Usado pela Ficha catalográfica
\usepackage{indentfirst}		% Indenta o primeiro parágrafo de cada seção.
\usepackage{color}		% Controle das cores
\usepackage{graphicx}			% Inclusão de gráficos
\usepackage{microtype} 			% para melhorias de justificação
\usepackage{diagbox} %tabelas com barras
% ---
		
% ---
% Pacotes adicionais, usados apenas no âmbito do Modelo Canônico do abnteX2
% ---
\usepackage{lipsum}				% para geração de dummy text
% ---

% ---
% Pacotes de citações
% ---
\usepackage[brazilian,hyperpageref]{backref}	 % Paginas com as citações na bibl
\usepackage[alf]{abntex2cite}	% Citações padrão ABNT

% --- 
% Pacotes matemáticos
% ---
\usepackage{amsthm}


% --- 
% CONFIGURAÇÕES DE PACOTES
% --- 
\newtheorem{definition}{Definição}
\newtheorem{teorema}{Teorema}
\newtheorem{corolario}{Corolário}
\graphicspath{{figuras/}}
% ---
\newtheorem{axiom}{Axioma}


\newcommand{\Mod}[1]{\ (\mathrm{mod}\ #1)}
%%%%%%%%%%%
\linespread{1.5} % espaçamento entre linhas
%%% Outros pacotes úteis - Igor - 05/11/2011
%%%%%%%%%%%%%%%%%%%%%%%%%%%%%%%%%%%%%%%%%%
%%% Insira aqui os pacotes necess�rios
%%%%%%%%%%%%%%%%%%%%%%%%%%%%%%%%%%%%%%
\usepackage{indentfirst}
\usepackage{array}

%%%%%%%%%%%%%%%%%%%%%%%%%%%%%%%%%%%%%%%%%%%%%%%%%%%%%%
%              Formata��o da P�gina                  %
%%%%%%%%%%%%%%%%%%%%%%%%%%%%%%%%%%%%%%%%%%%%%%%%%%%%%%

% horizontal
\setlength{\hoffset}{-1in}

\setlength{\oddsidemargin}{3.0cm}

\setlength{\textwidth}{160mm} % (210mm - 30mm - 20mm)

\setlength{\parindent}{1.25cm} % identa��o de cada par�grafo

% vertical
\setlength{\voffset}{-1in}
\addtolength{\voffset}{2.0cm}

\setlength{\topmargin}{0.0cm}

\setlength{\headheight}{5mm}
\setlength{\headsep}{5mm}

\setlength{\textheight}{247mm} % (297mm - 30mm - 20mm)

%\setlength{\footskip}{0mm}

%%%%%%%%%%%%%%%%%%%%%%%%%%%%%%%%%%%%%%%%%%%%%%%%%%%%%%


\begin{document}


%%%%%%%%%%%%%%%%%%%%%%%%%%%%%%%%%%%%%%%%%%%%%%%%%%%%%%
%                  Capa da Monografia                %
%%%%%%%%%%%%%%%%%%%%%%%%%%%%%%%%%%%%%%%%%%%%%%%%%%%%%%

\begin{titlepage}
  \begin{center}
    \Large{\textsc{Universidade Federal Fluminense} \\
           \textsc{Instituto de Computação} \\
           \textsc{Departamento de Ciência da Computação}
          }
    \par\vspace{3.0cm}
    \LARGE{Matheus Souza D'Andrea Alves}
%% Descomentar caso tenha outro aluno
     \par\vspace{3.0cm}
    \bigskip
    \LARGE{COLORAÇÃO DE $GRAFOS(r,\ell)$}
    \par\vfill
    \Large{Niterói-RJ\\2017}
  \end{center}
\end{titlepage}



%%%%%%%%%%%%%%%%%%%%%%%%%%%%%%%%%%%%%%%%%%%%%%%%%%%%%%
%                Numeracao em romano                 %
%%%%%%%%%%%%%%%%%%%%%%%%%%%%%%%%%%%%%%%%%%%%%%%%%%%%%%

\pagenumbering{roman}
\setcounter{page}{2}


%%%%%%%%%%%%%%%%%%%%%%%%%%%%%%%%%%%%%%%%%%%%%%%%%%%%%%
%                   Folha de Rosto                   %
%%%%%%%%%%%%%%%%%%%%%%%%%%%%%%%%%%%%%%%%%%%%%%%%%%%%%%


\begin{center}
Matheus Souza D'Andrea Alves


\vfill

COLORAÇÃO DE $GRAFOS(r,\ell)$

\vspace{3.0cm}

\begin{flushright}
\begin{minipage}{0.50\textwidth}

Trabalho submetido ao Curso de \linebreak Bacharelado em Ciência da
Computação
da Universidade Federal Fluminense como
requisito parcial para a obtenção do título de Bacharel em Ciência da
Computação.

\end{minipage}
\end{flushright}

\vspace{3.0cm}

\begin{flushleft}
Orientador: Dr. Uéverton dos Santos Souza 
\end{flushleft}

\vfill

Niterói-RJ\\2017

\end{center}

\newpage


\newpage

%%%%%%%%%%%%%%%%%%%%%%%%%%%%%%%%%%%%%%%%%%%%%%%%%%%%%%%%
%                  Dedicat�ria                                 %
%%%%%%%%%%%%%%%%%%%%%%%%%%%%%%%%%%%%%%%%%%%%%%%%%%%%%%%%

\begin{flushright}
\begin{minipage}{0.5\textwidth}

\vspace{15.0cm}
% espa�o do topo at� o in�cio da dedicat�ria



\textit{"Ignorance more frequently begets confidence than does knowledge:\\
		It is those who know little, and not those who know much,\\ who so positively assert that this or that problem\\ will never be solved by science.\\
		Charles Darwin (The Descent of Man - pg 3)}
\end{minipage}
\end{flushright}

%%%%%%%%%%%%%%%%%%%%%%%%%%%%%%%%%%%%%%%%%%%%%%%%%%%%%%%%
%                 Agradecimentos                             %
%%%%%%%%%%%%%%%%%%%%%%%%%%%%%%%%%%%%%%%%%%%%%%%%%%%%%%%%

\chapter*{Agradecimentos}

\thispagestyle{myheadings}

\noindent

[PLACE HOLDER]



%%%%%%%%%%%%%%%%%%%%%%%%%%%%%%%%%%%%%%%%%%%%%%%%%%%%
%            Resumo na l�ngua vern�cula            %
%%%%%%%%%%%%%%%%%%%%%%%%%%%%%%%%%%%%%%%%%%%%%%%%%%%%

\chapter*{Resumo}
\addcontentsline{toc}{chapter}{Resumo}

\thispagestyle{myheadings}

Um problema clássico na literatura é o problema de coloração própria de um grafo, isto é, encontrar uma q-coloração para um grafo G tal que todo vértice $v \in V(G)$ não possua nenhum vizinho da mesma cor e q seja mínimo. Esse problema é conhecido ser NP-Difícil para grafos gerais. O trabalho a seguir tem como proposta desvendar e catalogar a complexidade clássica e parametrizada de tal problema para a classe de Grafos$(r,\ell)$, i.e. grafos particionáveis em r conjuntos independentes e l cliques; Identificando as características que tornam o problema difícil e a relação do problema de coloração com outros problemas, quando abordado pela perspectiva parametrizada.

\bigskip
%

\noindent Palavras-chave: \textit{Complexidade parametrizada. Grafos$(r,\ell)$. Partição de grafos. Coloração de Grafos}



\chapter*{Abstract}
\addcontentsline{toc}{chapter}{Abstract}

\thispagestyle{myheadings}
\nocite{*}

A classical problem in the literature is the problem of proper coloring a graph, i.e. to find a q-coloring for a graph G such that every vertex $ v \in V (G) $ does not have any neighbor of the same color and q is the smallest possible number, a problem known to be NP-Hard for a general graphs. The following work attempts to uncover and catalog the parametrized complexity of such problem for the class of graphs$(r, \ell)$, i.e. partitionable graphs in r independent sets and l cliques; Identifying the characteristics that make the problem hard and the relation of the stated problem to other problems when approached by the parameterized perspective.

\bigskip
%

\noindent Keywords: \textit{Parametrized Complexity. Graph$(r,\ell)$. Graph Partitioning. Graph Coloring. }

\tableofcontents

\thispagestyle{myheadings}

\listoffigures
\addcontentsline{toc}{chapter}{Lista de Figuras}

\thispagestyle{myheadings}

\listoftables
\addcontentsline{toc}{chapter}{Lista de Tabelas}

\thispagestyle{myheadings}

\pagebreak
\pagenumbering{arabic}

\chapter{Introdução} \label{cap:introducao}s
\include{capitulos/analiseClassicaColoracao}
\include{capitulos/analiseParametrizada}
\chapter{Análise parametrizada de problemas relacionados a coloração}
\chapter{Resultados}
\chapter{Conclusão}

\cleardoublepage
\addcontentsline{toc}{chapter}{Referências Bibliográficas}
\bibliography{references}%{} % arquivos fonte com a bibliografia
\end{document}
