\chapter{Introdução} \label{cap:intro}

\section{Estruturas básicas}

\subsection{Grafos}
\begin{definition}
 Um Grafo $G$ é uma estrutura que contém um conjunto de vértices $V(G) = \{v_1,v_2,...,v_n\}$ e um conjunto de arestas $E(G)=\{e(v,j)\} | \{v,j\} \subset V(G)$, dizemos que o vértice $v$ pertence ao grafo $G$ se $v \in V(G)$, e que existe uma aresta entre $v$ e $u$ se $e(u,v) \in E(G)$, nesse trabalho usaremos apenas grafos não direcionados, dessa forma $e(u,v) \equiv e(v,u)$
\end{definition}

\subsection{Conjunto independente}
\begin{definition}
Um conjunto independente $r \subseteq G$ é uma partição de $G$ tal que: $ \{v,u\} \subset V(r) \implies \nexists e(v,u) \in E(r)$
\end{definition}

\subsection{Clique}
\begin{definition}
Uma clique $\ell \subseteq G$ é uma partição de $G$ tal que: $ \{v,u\} \subset V(\ell) \implies \exists e(v,u) \in E(\ell)$
\end{definition}

\subsection{Grafos$(r,\ell)$}
\begin{definition}
	Um Grafo dito Grafo$(r,\ell)$ ou abreviadamente $G(r,\ell)$ é qualquer grafo pertencente à classe dos grafos que podem ser particionados em r conjuntos independentes e l cliques.
\end{definition}

\subsection{Bipartição}
\begin{definition}
 Dizemos que um grafo $G$ possui uma bipartição quando todos os seus vértices podem ser divididos entre dois conjuntos independentes disjuntos.
\end{definition}

\section{Problemas mencionados}

\subsection{Coloração mínima de Grafos}
\begin{definition}
	Entrada: um Grafo $G$ e um inteiro $k$\\
	Questão: Cada vértice pertencente à $G$ pode ser colorido com uma entre $k$ cores
	de tal forma que dado quaisquer dois vértices adjacentes eles tenham cores distintas e $k$ seja o mínimo de cores possível?
\end{definition}

\subsection{Lista coloração de Grafos}
\begin{definition}
  Entrada: Uma paleta de cores $P$ e um Grafo $G$ onde todo $v \in V(G)$ pode ser colorido com um subconjunto $P(v) \subset P$\\
  Questão: É possível escolher uma cor dentro das de $P(v)$ para todo vértice $v$ de forma que dado quaisquer dois vértices adjacentes eles tenham cores distintas?
\end{definition}

\subsection{Clique multicolorida}
\begin{definition}
 Entrada: Um Grafo $G$ com uma k-coloração própria\\
 Questão: Existe em $G$ uma clique que contenha todas as k cores?
\end{definition}

\subsection{PreColoring extension}
\begin{definition}
 Entrada: Um grafo $G$ onde alguns vértices já possuem uma coloração definida com cores escolhidas dentre $k$ possíveis cores.
 Questão: É possível estender a coloração já existente para todo o grafo sem que dois vértices adjacentes possuam a mesma cor? 
\end{definition}

\subsection{Satisfabilidade Ponderada em Circuitos de Entrelaçamento $t$ e Profundidade $h$ \emph{WCS(t,h)}}
\begin{definition}
 Entrada um circuito de decisão $C$ com entrelaçamento $t$ e profundidade $h$\\
 Questão: $C$ possui uma atribuição satisfazível?
\end{definition}

\section{Complexidade clássica}
\subsection{Tratabilidade de tempo polinomial}
Um algoritmo de tempo polinomial é definido como um algoritmo cuja sua função de complexidade de tempo é $\mathcal{O}(p(n))$, para alguma função polinomial $p$, onde $n$ é usado para denotar o tamanho da entrada.

Um problema $\Pi$ pertence à classe $\P$ se e somente se $\Pi$ pode ser solucionado em tempo polinomial por algum algoritmo determinístico.

Um problema $\Pi$ pertence à classe $\textit{Np}$ se e somente se para um dado certificado há um algoritmo polinomial que o verifica sua validade.

\subsection{Reduções}
Dados dois problemas $\Pi$ e $\Pi'$ dizemos que $\Pi \propto \Pi'$ ($\Pi$ se reduz à $\Pi'$ em tempo polinomial) se existe um algoritmo capaz de construir uma instância $J$ de $\Pi'$ a partir de uma instância $I$ de $\Pi$ em tempo polinomial, tal que a partir de uma resposta para $J$ uma resposta para $I$ possa ser construída em tempo polinomial. 

\subsection{NP-Completude}
Um problema $\Pi'$ é dito $NP-Difícil$ se todo problema $\Pi \in NP$ se reduz à $\Pi'$, se $\Pi' \in NP$ então $\Pi'$ é $NP-Completo$.

\section{Complexidade parametrizada}

\subsection{Tratabilidade Parametrizada}
\begin{definition}
Dado um problema $\Pi$ e um conjunto de aspectos de $\Pi$ chamado $S = \{s_1,s_2,s_3,...,s_n\}$ denotamos por $\Pi(S)$ o problema $\Pi$ parametrizado por $S$.
\end{definition}
\begin{definition}
Dado um problema parametrizado $\Pi(S)$ dizemos que o mesmo é FPT(\emph{Fixed parameter tractable} (Tratado por parâmetro fixo)) se existe um algoritmo capaz de resolver $\Pi$ em $\mathcal{O}(f(S)\times n^c)$ onde $f(S)$ é uma função arbitrária e $c$ uma função $\mathcal{O}(1)$.
\end{definition}

\subsection{Intratabilidade Parametrizada}
Esta seção irá sumarizar as definições de $W-Hierarquia$ estabelecida por Downey et al.\cite{downey98}, para tanto observe as seguintes definições.
\begin{definition}
 Sejam $\Pi(k)$ e $\Pi'(k')$ onde $k' \leq g(k)$. Chamamos de FPT-redução de $\Pi(k)$ para $\Pi'(k')$ é uma transformação $R$ quando:
 \begin{itemize}
   \item $\forall x, x \in \Pi(k) \iff R(k) \in \Pi'(k')$
   \item $R$ é computável por um FPT-Algoritmo, com relação a $k$
 \end{itemize}
\end{definition}

\begin{definition}
 Um problema parametrizado $\Pi(k)$ pertence a classe $W[t]$ se e somente se existe uma FPT-Redução de tal problema para $WCS(t,h)$ para algum $h$ constante. Logo devido a transitividade de FPT-Redução, se existe uma FPT-Redução de qualquer problema $\Pi'(k')$ para $\Pi(k)$ então $\Pi(k) \in W[t]$
\end{definition}

\begin{definition}
 Um problema pertence a classe $Para-NP$ se existe um algoritmo $\mathcal{O}(f(k)g(n))$ onde $g(n)$ é uma função não-determinística.
\end{definition}

\begin{corolario}

\end{corolario}
