\chapter{Introdução} \label{cap:intro}

Uma das principais motivações do estudo de classes de grafos é o fato de que diversos problemas, que são difíceis para grafos em geral, tornam-se tratáveis quando restritos a classes especiais de grafos. Assim, busca-se delimitar a partir de que ponto um determinado problema pode ser resolvido de forma eficiente. 
Particularmente, o problema de particionamento em grafos tem despertado muito interesse devido às pesquisas de grafos perfeito e também pela procura de algoritmos eficientes para o reconhecimento de determinadas classes de grafos.
O problema de partição de grafos pode ser descrito como tendo por objetivo particionar o conjunto dos vértices de um grafo em subconjuntos $V_1,V_2, 
\ldots, V_k$ onde $V_1 \cup V_2 \cup \ldots \cup V_k = V$ e $V_i \cap V_j = \emptyset$, $i \neq j$, $1 \leq i \leq k$ e $1 \leq j \leq k$, exigindo-se, porém, algumas propriedades sobre estes subconjuntos de vértices. Estas propriedades podem ser \textit{internas}, como por exemplo exigir que os vértices de cada subconjunto $V_i$ sejam dois-a-dois adjacentes (isto é, $V_i$ seja uma clique) ou dois-a-dois não-adjacentes (isto é, $V_i$ seja um conjunto independente), ou \textit{externas}, onde as exigências são feitas sobre os pares de $V_i\times V_j$, podem ser adjacentes ou não-adjacentes entre si. 

Em outra perspectiva o problema de coloração de vértices de um grafo também é de grande interesse, tendo suas aplicações em diversas áreas; \emph{Pattern matching}, escalonamento em esportes e no trânsito, e resolução de problemas de \emph{Sudoku} são alguns exemplos de problemas onde a coloração de grafos pode ser empregada\cite{lewis15}. O problema de coloração mínima dos vértices de um grafo chamado também de coloração própria de um grafo é descrito como a atribuição de cores em um grafo $G$ de forma que seja possível atribuir a cada um dos vértices de $G$ uma entre $k$ cores de forma que, dado quaisquer dois vértices vizinhos em $G$ eles não compartilhem uma mesma cor e $k$ seja o menor número onde tal restrição é atingida.


O intuito deste trabalho é o de desvendar e classificar a dificuldade do problema de coloração em Grafos$(r,\ell)$. Ao nos aprofundar na investigação mostraremos ainda a proficuidade dos tamanhos das partições para a extração de um algoritmo FPT. 

O capítulo sobre análise computacional clássica se propõe a demonstrar uma dicotomia \emph{Polinomial / NP-Completo} para o problema abordando a quantidade de partições. Uma vez definida tal dicotomia o capitúlo de análise parametrizada busca através de parâmetros encontrar algoritmos tratáveis por parâmetro fixo para solucionar o problema em questão. Concluiremos o escrito sumarizando nossas descobertas e esclarecendo as repercursões de nosso trabalho e os novos desafios que nos pode trazer.

As seções mostradas a seguir se propõem a esclarecer as estruturas, problemas e ferramental utilizado durante o trabalho.

\section{Estruturas básicas}

 Um Grafo $G$ é uma estrutura que contém um conjunto de vértices $V(G) = \{v_1,v_2,...,v_n\}$ e um conjunto de arestas $E(G)=\{(v,j) | \{v,j\} \subset V(G)$, dizemos que o vértice $v$ pertence ao grafo $G$ se $v \in V(G)$, e que existe uma aresta entre $v$ e $u$ se $(u,v) \in E(G)$. Nesse trabalho usaremos apenas grafos não direcionados, dessa forma $(u,v) \equiv (v,u)$. Um Grafo dito Grafo$(r,\ell)$ ou abreviadamente $G(r,\ell)$ é qualquer grafo pertencente à classe dos grafos que podem ser particionados em $r$ conjuntos independentes e $\ell$ cliques.
 

Um conjunto independente $R \subseteq G$ é uma partição de $G$ tal que: $ \{v,u\} \subset V(R) \implies \nexists (v,u) \in E(R)$. Já uma clique $L \subseteq G$ é uma partição de $G$ tal que: $ \{v,u\} \subset V(L) \implies \exists (v,u) \in E(L)$. Dizemos que um grafo $G$ possui uma bipartição quando todos os seus vértices podem ser divididos entre dois conjuntos independentes disjuntos.

\section{Problemas abordados}


\begin{definition}
	Coloração mínima de Grafos.\\
	\par\addvspace{.1\baselineskip}
	\noindent
	\begin{tabularx}{\textwidth}{@{\hspace{\parindent}} l X c}
		\textbf{Entrada:} & um Grafo $G$ e um inteiro $k$.\\% Input
		\textbf{Pergunta:} & É possível atribuir a cada vértice pertencente à $G$ uma entre $k$ cores
	de tal forma que dado quaisquer dois vértices adjacentes eles tenham cores distintas e $k$ seja o mínimo de cores possível?
	\end{tabularx}
	\par\addvspace{.5\baselineskip}
\end{definition}
\pagebreak

\begin{definition}
	Lista coloração de Grafos.\\
	\par\addvspace{.1\baselineskip}
	\noindent
	\begin{tabularx}{\textwidth}{@{\hspace{\parindent}} l X c}
		\textbf{Entrada:} & Uma paleta de cores $P$ e um Grafo $G$ onde todo $v \in V(G)$ pode ser colorido com um subconjunto $P(v) \subset P$.\\% Input
		\textbf{Pergunta:} & É possível escolher uma cor dentro das de $P(v)$ para todo vértice $v$ de forma que dado quaisquer dois vértices adjacentes eles tenham cores distintas?
	\end{tabularx}
	\par\addvspace{.5\baselineskip}
\end{definition}


\section{Complexidade computacional}

Um algoritmo de tempo polinomial é definido como um algoritmo cuja sua função de complexidade de tempo é $\mathcal{O}(p(n))$, para alguma função polinomial $p$, onde $n$ é usado para denotar o tamanho da entrada.

Um problema $\Pi$ pertence à classe $\P$ se e somente se $\Pi$ pode ser solucionado em tempo polinomial por algum algoritmo determinístico. Um problema $\Pi$ pertence à classe $\textit{NP}$ se e somente se para um dado certificado há um algoritmo que verifica sua validade em tempo polinomial.

Dados dois problemas $\Pi$ e $\Pi'$ dizemos que $\Pi \propto \Pi'$ ($\Pi$ se reduz à $\Pi'$ em tempo polinomial) se existe um algoritmo capaz de construir uma instância $J$ de $\Pi'$ a partir de uma instância $I$ de $\Pi$ em tempo polinomial, tal que a partir de uma resposta para $J$ uma resposta para $I$ possa ser construída em tempo polinomial. 

\subsection{NP-Completude}
Um problema $\Pi'$ é dito $NP-Difícil$ se todo problema $\Pi \in NP$ se reduz à $\Pi'$. Se $\Pi' \in NP$ então $\Pi'$ é $NP-Completo$.

\section{Complexidade parametrizada}

\subsection{Tratabilidade Parametrizada}
\begin{definition}
Dado um problema $\Pi$ e um conjunto de aspectos de $\Pi$ chamado $S = \{s_1,s_2,s_3,...,s_n\}$ denotamos por $\Pi(S)$ o problema $\Pi$ parametrizado por $S$.
\end{definition}
\begin{definition}
Dado um problema parametrizado $\Pi(S)$ dizemos que o mesmo é FPT (\emph{Fixed parameter tractable} (Tratável por parâmetro fixo)) se existe um algoritmo capaz de resolver $\Pi$ em $\mathcal{O}(f(S)\times n^c)$ onde $f(S)$ é uma função arbitrária e $c$ uma função $\mathcal{O}(1)$.
\end{definition}

\subsection{Intratabilidade Parametrizada}

Esta seção irá sumarizar as definições de $W-hierarquia$ estabelecida por Downey et al.\cite{downey98}, para tanto observe as seguintes definições.

\begin{definition}\label{def:fpt-red}
 Sejam $\Pi(k)$ e $\Pi'(k')$ onde $k' \leq g(k)$. Chamamos de FPT-redução de $\Pi(k)$ para $\Pi'(k')$ é uma transformação $R$ quando:
 \begin{itemize}
   \item $\forall x, x \in \Pi(k) \iff R(k) \in \Pi'(k')$
   \item $R$ é computável por um FPT-Algoritmo, com relação a $k$
 \end{itemize}
\end{definition}

\begin{definition}\label{def:wcs}
	Satisfabilidade Ponderada em circuitos de entrelaçamento $t$ e profundidade $h$ $WCS(t,h)$.\\
	\noindent
	\begin{tabularx}{\textwidth}{@{\hspace{\parindent}} l X c}
		\textbf{Entrada:} & um circuito de decisão $C$ com entrelaçamento $t$ e profundidade $h$.\\% Input
		\textbf{Pergunta:} & $C$ possui uma atribuição satisfazível?
	\end{tabularx}
	\par\addvspace{.5\baselineskip}
\end{definition}

Usando as definições \ref{def:fpt-red} e \ref{def:wcs} definiremos então a pertinência de um problema à classe $W[t]$.

\begin{definition}
 Um problema parametrizado $\Pi(k)$ pertence a classe $W[t]$ se e somente se existe uma FPT-Redução de tal problema para $WCS(t,h)$ para algum $h$ constante. Logo devido a transitividade de FPT-Redução, se existe uma FPT-Redução de qualquer problema $\Pi'(k')$ para $\Pi(k)$ então $\Pi(k) \in W[t]$
\end{definition}

