\chapter{Introdução} \label{cap:intro}

\section{Conceitos básicos}

\subsection{Grafos$(r,\ell)$}
\begin{definition}
	Um Grafo dito Grafo$(r,\ell)$ ou abreviadamente $G(r,\ell)$ é qualquer grafo pertencente á classe dos grafos que podem ser particionados em r conjuntos independentes e l cliques.
\end{definition}

\subsection{Coloração mínima de Grafos}
\begin{definition}
	Entrada: um Grafo $G$ e um inteiro $k$\\
	Questão: Cada vértice pertencente à $G$ pode ser colorido com uma entre $k$ cores
	de tal forma que dado quaisquer dois vértices adjacentes eles tenham cores distintas e $k$ seja o mínimo de cores possível?
\end{definition}

\subsection{Lista coloração de Grafos}
\begin{definition}
  Entrada: Uma paleta de cores $P$ e um Grafo $G$ onde todo $v \in V(G)$ pode ser colorido com um subconjunto $P(v) \subset P$\\
  Questão: É possível escolher uma cor dentro das de $P(v)$ para todo vértice $v$ de forma que dado quaisquer dois vértices adjacentes eles tenham cores distintas?
\end{definition}

\subsection{Clique multicolorida}
\begin{definition}
 Entrada: Um Grafo $G$ com uma k-coloração própria\\
 Questão: Existe em $G$ uma clique que contenha todas as k cores?
\end{definition}
