\chapter{Análise clássica para coloração em Grafos$(r,\ell)$}

O problema de coloração aplicado a Grafos$(r,\ell)$ pode ser facilmente solucionável em alguns casos, por exemplo, um grafo sem arestas i.e um Grafo(1,0) é colorível com apenas uma cor, um grafo completo, ou seja um Grafo(0,1), é colorível com $n$ cores onde $n$ é a quantidade de vértices desse grafo, um grafo split, por sua vez, é um Grafo(1,1) e pode ser colorido utilizando $k$ cores, onde $k$ é o tamanho da clique máxima do grafo, pois assumindo uma partição (1,1) onde a clique é máxima, cada vértice do conjunto independente pode ser colorível com alguma cor já presente na clique.
E por fim, grafos bipartidos são coloridos com 2 cores uma cor para cada conjunto independente.

Sendo assim, sabemos que coloração é um problema solucionável em tempo polinomial em grafos completos, nulos, splits e bipartidos. Portanto, temos como ponto de partida para a nossa análise as questões apresentadas na Tabela \ref{tab:tabela_part1dictrl}.

%primeira tabela de Dicotomia $\P/\NPc$
\begin{table}[!htb]
	\center
	\begin{tabular}{l|*{7}c}
		\toprule
		\backslashbox{$r$}{$\ell$} & 0 & 1 & 2 & 3 & 4 & \ldots & n\\
		\midrule
        0 & \P & \P & \? & \? & \? & \ldots & \?\\
        1 & \P & \P & \? & \? & \? & \ldots & \?\\
        2 & \P & \? & \? & \? & \? & \ldots & \?\\
        3 & \? & \? & \? & \? & \? & \ldots & \?\\
        4 & \? & \? & \? & \? & \? & \ldots & \?\\
        $\vdots$ & $\vdots$ & $\vdots$ & $\vdots$ & $\vdots$ & $\vdots$ & $\ddots$ & \?\\
        n & \? & \? & \? & \? & \? & \ldots & \?\\
    \bottomrule
	\end{tabular}%
	\caption{1ª Dicotomia $\P/\NPc$ parcial do problema de coloração em Grafos$(r,\ell)$}
	\label{tab:tabela_part1dictrl}%
\end{table}%

Neste ponto, inciaremos a nossa discussão sobre a complexidade do problema de coloração para as demais classes de grafos $(r,\ell)$. 
 
	\begin{teorema}
		Coloração de Grafo(0,2) pode ser solucionado em tempo polinomial.
	\end{teorema}
	\begin{proof}
		Um Grafo(0,2) $G$, também conhecido como grafo co-bipartido, é um grafo onde seu conjunto de vértices pode ser particionado em dois conjuntos $K_1$ e $K_2$ tal que $V(G)=K_1\cup K_2$, $K_1\cap K_2=\emptyset$ e $G[K_1]$, $G[K_2]$ são grafos completos.
É bem conhecido na literatura que um grafo co-bipartido é perfeito~\cite{bollo98}, consequentemente, seu número cromático é igual ao tamanho da maior clique. Note que encontrar a clique máxima de um grafo co-bipartido é equivalente a encontrar o conjunto independente máximo do complemento de $G$ que, por sua vez, pode ser feito em tempo polinomial através de um algoritmo para Cobertura por Vértices em grafos bipartidos~\cite{konig31,zing12}. Portanto, ao encontrar a cobertura por vértices mínima para o complemento de $G$ encontramos a clique máxima de $G$, e como $G$ é um grafo perfeito, sabemos resolver o problema de coloração em tempo polinomial.
	\end{proof}
	
A seguir apresentamos outra classe em que o problema pode ser solucionado em tempo polinomial.	

\begin{teorema}
Coloração de Grafo(3,0) é Polinomial.
\end{teorema}

\begin{proof}
Se $G$ é um grafo que admite um partição (3,0), sabe-se que o número cromático de $G$ é no máximo 3, sendo assim, para determinarmos o valor exato de seu número cromático, basta verificar se $G$ é um grafo vazio (número cromático igual a um), e caso contrário verificar se $G$ é bipartido (número cromático igual a dois), como ambas verificações são executáveis em tempo polinomial podemos afirmar que o número cromático de um grafo(3,0) pode ser determinado em tempo polinomial.
	\end{proof}
	
Como o número cromático de grafos(3,0) pode ser solucionável em tempo polinomial, uma pergunta interessante passa a ser a complexidade de determinar o número cromático de um grafo (4,0).	
	

\begin{teorema}
	Coloração de Grafo(4,0) é NP-Completo.
\end{teorema}
	
\begin{proof}
Sabemos que todo grafo planar é 4-colorível \cite{appel77}, logo os grafos planares formam uma subclasse dos grafos(4,0). Portanto, para determinar o número cromático de um grafo planar $G$ é necessário solucionar 3-coloração em grafos planares, no entanto, tal problema é NP-Completo[???], logo descobrir o número cromático de um grafo planar $G$ é NP-Completo e consequentemente resolver coloração em grafos(4,0) é NP-Completo.
	\end{proof}

É importante notar que, todo $Grafo(r,\ell)$ é simultaneamente um $Grafo(r,\ell+1)$ e um $Grafo(r+1,\ell)$ já que por definição algumas partes podem ser vazias, portanto se o problema de coloração é NP-Completo para Grafo$(r,\ell)$ então ele é NP-Completo para $Grafo(r+1,\ell)$ e $Grafo(r,\ell+1)$.

Os resultados anteriores e a monotonicidade observada anteriormente no leva ao seguinte cenário: 

%Segunda tabela de Dicotomia $\P/\NPc$
\begin{table}[htb!]
	\center
	\begin{tabular}{l|*{7}c}
		\toprule
		\backslashbox{$r$}{$\ell$} & 0 & 1 & 2 & 3 & 4 & \ldots & n\\
		\midrule
            0 & \P & \P & \P & \? & \? & \ldots & \?\\
            1 & \P & \P & \? & \? & \? & \ldots & \?\\
            2 & \P & \? & \? & \? & \? & \ldots & \?\\
            3 & \P & \? & \? & \? & \? & \ldots & \?\\
            4 & \NPc & \NPc & \NPc & \NPc & \NPc & \ldots & \NPc\\
            $\vdots$ & $\vdots$ & $\vdots$ & $\vdots$ & $\vdots$ & $\vdots$ & $\ddots$ & \NPc\\
            n & \NPc & \NPc & \NPc & \NPc & \NPc & \ldots & \NPc\\
            \bottomrule
	\end{tabular}%
	\caption{2ª Dicotomia $\P/\NPc$ parcial do problema de coloração em Grafos$(r,\ell)$}
	\label{tab:tabela_part2dictrl}%
\end{table}%

Como podemos observar, ainda nos falta mostrar a complexidade para alguns casos de fronteira, que necessitam de uma demonstração mais complexa.
%Demonstrar lista-coloração(r,l)Npc -> coloração(r,l+1)Npc e seus colorários (1,2) & (2,1)

Nesto ponto, iremos apresentar uma ferramenta importante que nos auxiliará nos demais resultados.
 
\begin{teorema}
\label{theorem:list-coloring}
O problema de lista coloração de um grafo $(r,\ell)$, se reduz ao problema de coloração de um grafo $(r,\ell+1)$.
\end{teorema}

\begin{proof}
Seja $G$ uma instância do problema de lista coloração em grafos$(r,\ell)$, isto é, cada vértice $v \in V(G)$ possui uma lista de cores $S_v$. Seja $C=\{c_1,c_2,\ldots,c_k\}$ a paleta de cores utilizada em $G$. A  partir de $G$ criaremos um grafo $H_G$, instância do problema de coloração, da seguinte forma:

\begin{itemize}
\item inicialmente faça $V(H_G)=V(G)$ e $E(H_G)=E(G)$;
\item adicione em $H_G$ uma clique $K$ formada por $k$ novos vértices, $w_1,w_2,\ldots,w_k$, cada vértice $w_i \in V(K)$ representará a cor $c_i$ presente em $C$;
\item para todo vértice $v \in V(H_G)\cap V(G)$ e todo vértice $w_i \in V(K)$ adicione uma aresta $(u_i,v)$ em $H_G$ se $v$ não possui a cor $c_i$ na lista coloração $S_v$ em $G$.
\end{itemize}

Neste ponto nos resta demonstrar que $G$ possui uma lista coloração própria se e somente se $H_G$ é k-colorível, onde para $k$ é o número de cores da paleta $C$ de $G$.
	
%------	

Suponha que $G$ possui uma lista coloração. Iremos mostrar que $H_G$ admite um $k$-coloração como segue. Atribua para cada vértice em $V(H_G)\cap V(G)$ a mesma cor que lhe foi atribuída em $G$. Note que a clique $K$ possui exatamente $k$ vértices, consequentemente para colorirmos $K$ precisaremos de $k$ cores, assumimos que $w_1$ será colorido com $c_1$, $w_2$ será colorido com $c_2$ e assim por diante. Por construção, só existe aresta de $w_i$ para um vértice $v \in V(H_G)\setminus V(K)$ se $v$ não possui $c_i$ em sua lista de cores, portanto a coloração atribuída a clique $K$ não conflita com as cores atribuídas aos vértices em $V(H_G)\cap V(G)$, como a paleta de $G$ possui $k$ cores, temos que $H_G$ é $k$-colorível.
	

Suponha que o grafo $H_G$ possua uma $k$-coloração. Por construção, $k$ é a cardinalidade de $K$, observe que a remoção de K não afeta a coloração de $H_G \setminus K$.
Como $H_G$ é $k$-colorível e a clique $K$ possui $k$ vértices todas as cores de tal $k$-coloração estão presentes em $K$. Sem perda de generalidade, podemos assumir que as cores $c_1,c_2,...,c_k$ estão atribuídas aos vértices $w_1,w_2,...,w_k$, respectivamente.
Por construção de $H_G$ todo par $v,w_i$ onde $v \in V(H_G) \setminus V(K)$ e $w_i \in V(K)$ é não adjacente se e somente se o vértice $v$ não possui $c_i$ em sua lista coloração no grafo $G$, logo a coloração atribuídas aos vértices em $H_G \setminus K$ formam uma solução para lista coloração em $G$. Portanto $G$ é uma instância sim de lista coloração.
\end{proof}


A partir do Teorema~\ref{theorem:list-coloring}, obtemos os seguintes corolários:

\begin{corolario}
O problema de coloração é NP-Completo para Grafos$(1,2)$.
\end{corolario}

\begin{proof}
Segue da NP-Completude de lista coloração em Grafos$(1,1)$, demonstrado por Jansen et al.\cite{jansen1997}.
\end{proof}

	
\begin{corolario}
O problema de coloração é NP-Completo em Grafos$(2,1)$.
\end{corolario}    

\begin{proof}
Segue da NP-Completude de lista coloração em grafos bipartido demonstrado por Fellows et al.\cite{fellows07}.
\end{proof}

Neste ponto, nos resta determinar a complexidade coloração em grafos (0,2).


\begin{teorema} \label{teorema:lista-2}
Lista coloração é NP-Completo para Grafos(0,2).
\end{teorema}

\begin{proof}
		Para essa demonstração nos basearemos em um resultado obtido por Jansen \cite{jansen1999}. A demonstração se baseia em realizar uma redução do problema 3-SAT restrito para lista coloração de co-bipartido i.e. Grafo(0,2).
		Suponha o problema 3-SAT com as seguintes restrições:
		\begin{itemize}
			\item cada cláusula $c_i$ contém dois ou três terminais.
			\item cada literal ou sua negação aparece no máximo em 3 cláusulas
		\end{itemize}
		Construiremos agora uma instância de lista coloração da seguinte forma:\newline
		Para cada literal $j$ crie seis vértices:
		$a_j^{(1)}$, $a_j^{(2)}$, $a_j^{(3)}$;
		$b_j^{(1)}$, $b_j^{(2)}$, $b_j^{(3)}$. Atribuindo a cada vértice uma lista de cores da seguinte forma:\newline
		$a_j^{(k)}$ <= \{$x_j^{(k)}$, $\overline{x_j}^{(k)}$ \}; $b_j^{(k)}$ <= \{$\overline{x_j}^{(k)}$,$x_j^{((k \Mod{3}) + 1 )}$ \}\newline
		Definimos como A o conjunto de todos os $a_j^{(k)}$ e B o conjunto de todos os $b_j^{(k)}$ e construímos uma clique com os vértices de A e B. Observe que só existem duas maneiras de se colorir este grafo:
		\begin{itemize}
			\item (1)  $f(a_j^{(k)}) = x_j^{(k)} => b_j^{(k)} = \overline{x_j}^{(k)}$
			\item (2)  $f(a_j^{(k)}) = \overline{x_j}^{(k)} => b_j^{(k)} = x_j^{((k \Mod{3}) + 1 )}$
		\end{itemize}
		Agora, para cada cláusula definimos um vértice $c_i$ e sua lista de cores da seguinte forma: para cada literal $j$ ou sua negação $\overline{j}$ presente na cláusula adicionamos à lista de $c_i$ o $x_j^{(k)}$ onde k é o índice de ocorrência do literal ou de sua negação.
		
		Por exemplo, suponha o seguinte 3-SAT:
		
		$(p \lor q \lor r) \land (\neg{p} \lor q \lor r) \land (\neg{p} \lor \neg{r} \lor s)$
		
		suas cláusulas seriam traduzidas para
		\begin{itemize}
			\item $c_1$ com lista: \{$p^1$, $q^1$, $r^1$ \}
			\item $c_2$ com lista: \{$\overline{p}^2$, $q^2$, $r^2$ \}
			\item $c_3$ com lista: \{$\overline{p}^3$, $\overline{r}^3$, $s^1$ \}
		\end{itemize}
		Seja C o conjunto contendo todos os $c_i$ criamos uma clique com $C \cup A$.
		Nosso grafo tem portanto a seguinte configuração(considere $x'$ como $\overline{x}$):
		\begin{figure}[!ht]
			\centering
			\includesvg{3-SAT.svg}
			\caption{Grafo G: Transformação de 3-SAT em co-bipartido com foco na cláusula P }
		\end{figure}
		
		Suponha a cláusula p, se p é verdadeiro então $a_p^{(1)},a_p^{(2)},a_p^{(3)}$ será colorido com $p'^1,p'^2,p'^3$, permitindo que a cor $p^x$ possa, e que a cor $p'^x$ não possa ser escolhidas para colorir uma cláusula.
		
		De tal forma, podemos facilmente notar que, expandindo a explicação anterior para os outros terminais uma resposta sim para o problema 3-SAT restrito nos leva a uma solução do problema de lista coloração em co-bipartido por exclusão das cores nas listas disponíveis. Em contrapartida a existência de uma lista coloração válida para o co-bipartido mostra uma solução para o 3-SAT restrito correspondente simplesmente descobrindo a representação em valor de literal das cores escolhidas para as cláusulas. 	
\end{proof}

\begin{corolario}
O problema de coloração é NP-Completo em Grafos$(0,3)$.
\end{corolario}

\begin{proof}
 Tal resultado sai diretamente da prova dos Teoremas \ref{theorem:list-coloring} e \ref{teorema:lista-2}.
 \end{proof}

Portanto temos a dicotomia $\P/\NPc$ do problema de coloração em Grafos$(r,\ell)$.
\newpage
\begin{table}[!h]
	\center
	\begin{tabular}{l|*{7}c}
		\toprule
		\backslashbox{$r$}{$\ell$} & 0 & 1 & 2 & 3 & 4 & \ldots & n\\
		\midrule
		0 & \P & \P & \P & \NPc & \NPc & \ldots & \NPc\\
		1 & \P & \P & \NPc & \NPc & \NPc & \ldots & \NPc\\
		2 & \P & \NPc & \NPc & \NPc & \NPc & \ldots & \NPc\\
		3 & \P & \NPc & \NPc & \NPc & \NPc & \ldots & \NPc\\
		4 & \NPc & \NPc & \NPc & \NPc & \NPc & \ldots & \NPc\\
		$\vdots$ & $\vdots$ & $\vdots$ & $\vdots$ & $\vdots$ & $\vdots$ & $\ddots$ & \NPc\\
		n & \NPc & \NPc & \NPc & \NPc & \NPc & \ldots & \NPc\\
		\bottomrule
	\end{tabular}%
	\caption{Dicotomia $\P/\NPc$ do problema de coloração em Grafos$(r,\ell)$}
	\label{tab:tabela_dictrl}%
\end{table}%
