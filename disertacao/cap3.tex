\chapter{Análise parametrizada para coloração em Grafos(2,1)}
Tendo mostrado a complexidade clássica nos é interessante agora que elucidemos quais caractéristicas dos grafos$(r,\ell)$ se mostram propícias a abordagem parametrizada, a cardinalidade de suas partições se mostrou uma interessante característica.
Decidimos abordar a classe (2,1), já que a mesma é a classe onde o problema é NP-Completo com o menor número de partições.

Um Grafo(2,1) é um grafo particionado em 2 conjuntos independentes e 1 clique, portanto ele nos entrega 3 naturais candidatos a parametrização, o tamanho da clique $\ell$, o tamanho do menor conjunto independente $r_1$ e o tamanho do maior conjunto independente $r_2$.

\section{Parametrização pelo tamanho do menor conjunto independente}
Em \cite{fellows07} Fellows (et. al) mostrou que o problema de lista coloração é $W[1]-hard$ parametrizado pela treewidth através da transformação do problema da clique multicolorida parametrizada pelo tamanho da clique para tal, nos aproveitaremos dessa transformação para mostrar que:
\begin{teorema}
	Coloração em Grafos(2,1) é $W[1]-hard$ quando parametrizado pelo tamanho do menor conjunto independente.
\end{teorema}
\begin{proof}
	Observe a seguinte transformação.
	
	O problema da clique multicolorida é conhecidamente $W[1]-hard$.
	
	Portanto suponha tal $G$ proposto ao problema de clique multicolorida, temos como intenção montar um problema de lista coloração em um grafo $G'$ a partir dele, para tanto seguimos os seguintes passos:
	\begin{itemize}
		\item Para cada cor $i$ presente em $G$ cria-se em $G'$ um vértice $v_i$ (os chamaremos de vértices-cor).
		\item Para cada vértice $u$ em $G$ colorido com a cor $i$, adicionamos à lista do vértice-cor $v_i$ em $G'$ uma cor $c_u$ relacionada a esse vértice (as chamaremos de cores-vértice).
		\item Para cada aresta $e(x,y) \notin E(G)$ onde $x,y \in V(G)$ cria-se em $G'$ um vértice $z_e$ adjacente ao vértice-cor $v_i$ onde $i$ representa as cores de $x$ e $y$, a lista coloração de $z_e$ será formada por $c_x$ e $c_y$.
	\end{itemize}
	É notável que a treewidth de $G'$ é dada por $k$, já que a remoção dos vértices-cor leva a um grafo sem arestas. Assim sendo se $G$ possui uma clique multicolorida podemos facilmente colorir $G'$ da seguinte forma:
	
	Ao vértice-cor $v_i$ atribua a cor-vértice $c_u$ onde $u$ é o vértice colorido com a cor $i$ em $G$. Dessa forma todos os vértices $z_e$ possuem ainda uma cor disponível para sua coloração já que ele representa uma não-aresta em $G$. 
	
	Para a volta observe que uma lista coloração válida em $G'$ implica em uma clique multicolorida em $G$, isso se dá pois dois vértices $x,y$ coloridos com cores diferentes em $G$ não aparecem em uma lista de algum $z_e$ em $G'$ se e somente se existe uma aresta $e(x,y) \in E(G)$, portanto as cores-vértices escolhidas para os vértices $v_i$ são uma respectivamente uma clique formadas por tais $i$ em $G$. Mostramos assim que lista coloração parametrizada por treewidth é $W[1]-hard$.
	
	Sabemos que coloração em Grafos(2,1) é equivalente a lista coloração em um grafo bipartido, portanto nossa tentativa de parametrizar a coloração de (2,1) pelo tamanho do menor conjunto independente é equivalente a parametrizar lista-coloração em bipartidos pelo tamanho do menor conjunto independente, é de pouca dificuldade ver que a treewidth de um grafo bipartido existe em função do menor independente, mostrando assim que coloração em Grafos(2,1) parametrizada pelo tamanho do menor conjunto independente é $W[1]-hard$. 
	
\end{proof}

\section{Parametrização pelo tamanho do maior independente}

 Sabemos agora que a parametrização pelo menor independente não nos traz um algoritmo FTP, porém ao analisarmos o comportamento do problema quando parametrizado pelo maior independente vemos que a limitação do tamanho de $r2$ também limita $r1$; Tendo tal limitação a utilização do método de árvore de altura limitada se mostra uma abordagem válida, como mostra o seguinte teorema.
\begin{teorema}
  Coloração de Grafos(2,1) é FTP quando parametrizado pelo tamanho do maior conjunto independente.
\end{teorema}
\begin{proof}
  Para tal demonstração onde $k$ é o tamanho de $r2$, observe que são necessárias pelo menos $t$ cores, onde $t$ é a cardinalidade da clique para colorir tal grafo, novamente usaremos a estratégia de transformar coloração de (2,1) em lista coloração de bipartido.
  
  Em uma lista coloração de bipartido, se um vértice possui uma lista com mais cores do que o tamanho de sua vizinhança, ele sempre terá disponível uma cor para sua coloração, podemos portanto remover esse vértice do grafo sem alterar sua coloração, ao chegarmos ao ponto onde todo vértice com tal configuração foi removido temos que $t$ está limitado em função de $k$, portanto rodar um algoritmo de força bruta para encontrar a coloração se mostra FTP.     
\end{proof}

\section{Parametrização pelo tamanho da clique}

Para a demonstração da complexidade parametrizada utilizando $k=\#\ell$ nos voltamos novamente para transformação da clique em um Grafo(2,1) em listas coloração do restante bipartido, dessa forma nosso problema parametrizado original se torna um novo problema, lista coloração de bipartido parametrizado pelo tamanho da paleta de cores. 

Mostraremos no entanto que essa parametrização não é proveitosa já que o problema se mostra equivalente à PreColoring Extension com limite de cores, mostrado ser NP-Completo para grafos bipartidos mesmo quando sua paleta é de tamanho 3\cite{kratochvil94}.

\begin{teorema}
	Lista coloração em bipartidos é NP-Completo
\end{teorema}
\label{theorem:list-coloring-bipartide}
\begin{proof}
	Suponha uma instância $P$ do problema PreColoring Extension e $G$ seu grafo de entrada, sabemos que $G$ possui uma paleta $C$ de cores de tamanho definido, e que existem $v \in V(G)$ que já estão coloridos com uma cor $c \in C$, podemos ver tal configuração como um grafo $G'$ onde os vértices $v$ possuem listas contendo apenas $c$, e os demais vértices possuem listas de tamanho $\#C$ contendo todas as cores, nos levando a um problema de lista coloração $Q$ que tem como entrada $G'$.
	
	Uma coloração possível para $G$ implica em uma coloração possível para $G'$, já que nos basta atribuir aos vértices em $G'$ as mesmas cores atribuídas em $G$. De forma análoga, uma lista coloração possível em $G'$ implica em uma coloração possível em $G$.
\end{proof}


Apesar do tamanho da paleta não ter se mostrado uma escolha adequada, ele levanta novos parametros que são interessantes para o problema de lista coloração em bipartidos, observe pois que, sabemos que Precoloring extension é polinomial se todas as listas tem tamanho 1 ou 2 \cite{hujter93}, e NP-Completo se todas tem listas e tamanho 1 à 3 \cite{kratochvil94}, isso levanta duas formas de se abordar o problema, o que acontece quando o número de vértices com listas de tamanho 1 e 2 varia, e o que acontece quando o número de vertices com listas de tamanho 3 varia.

Mostraremos nas próximas seções como se dão tais comportamentos e como eles se relacionam a coloração de Grafos$(r,\ell)$

\section{Parametrizado pela quantidade de vértices vizinhos à clique}
Nos focaremos nessa seção em grafos(2,1) cuja a clique tenha tamanho 3, já mostrada ser o menor tamanho necessário para que o problema de seja NP-Completo mostrado no teorema \ref{theorem:list-coloring-bipartide} e em \cite{kratochvil94}. Portanto um vértice que é vizinho da clique tem necessariamente uma lista contendo uma ou duas cores, já que um vértice não pertencente a clique que tenha lista de tamanho zero deveria fazer parte da clique, em contrapartida um vértice com lista tamanho 3 é um vértice não vizinho a clique. 

Mostraremos que mesmo quando parametrizado pela quantidade de vértices com listas de tamanho um, dois, ou um e dois o problema é Para-Np-completo. Para tanto é necessário encontrar uma instância do problema já parametrizado cuja solução permanece igualmente difícil.

Portanto essa seção será dividida em três casos, um contendo vértices de listas tamanho um, outro contendo vértices com listas de tamanhos dois, e finalmente contendo listas de tamanho um e dois.
\subsection{Apenas vértices com listas tamanho um}
É importante ressaltar que os seguintes teoremas estabelecem a base para a resolução do problema envolvendo os vizinhos da clique. 
\begin{teorema}
  Seis vértices com lista de tamanho um são necessários para que lista coloração em bipartido seja Para-NP-completo. 
\end{teorema}
\begin{proof}
 Sabemos que em nosso problema temos dois conjuntos independentes, $r1$ e $r2$, também é verdade que exceto pelos citados seis vértices todos os outros vértices tem listas de tamanho três, os vértices de $r1$ podem estar ligados arbitrariamente aos vértices de $r2$. 
\begin{figure}[!ht]
		\centering
		\includesvg{seis-vertices-lista-um.svg}
		\caption{Stuff }
\end{figure}
\end{proof}
\begin{teorema}
  Três vértices com lista de tamanho 3 são suficientes para que lista coloração em bipartido seja Para-NP-completo.
\end{teorema}
\begin{proof}
\end{proof}
 Dado os resultados mostrados nessa seção mostramos que vértices com listas tamanho um não é um parâmetro viável para uma solução FPT. 
\subsection{Vértices com listas de tamanho dois}
Mostraremos nessa seção que vértices com listas tamanho dois não são suficientes para que um algoritmo FPT seja extraído

\subsection{Vértices com listas de tamanho um e dois}
