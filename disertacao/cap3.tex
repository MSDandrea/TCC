\chapter{Análise parametrizada para coloração em Grafos(2,1)}
Tendo mostrado a complexidade clássica nos é interessante agora que elucidemos quais caractéristicas dos grafos$(r,\ell)$ se mostram propícias a abordagem parametrizada, a cardinalidade de suas partições se mostrou uma interessante característica.
Decidimos abordar a classe (2,1), já que a mesma é a classe onde o problema é NP-Completo com o menor número de partições.

Um Grafo(2,1) é um grafo particionado em 2 conjuntos independentes e 1 clique, portanto ele nos entrega 3 naturais candidatos a parametrização, o tamanho da clique $\ell$, o tamanho do menor conjunto independente $r_1$ e o tamanho do maior conjunto independente $r_2$.

\section{Parametrização pelo tamanho do menor conjunto independente}
Em \cite{fellows07} Fellows (et. al) mostrou que o problema de lista coloração é $W[1]-difícil$ parametrizado pela treewidth através da transformação do problema da clique multicolorida parametrizada pelo tamanho da clique para tal, nos aproveitaremos dessa transformação para mostrar que:
\begin{teorema}
	Coloração em Grafos(2,1) é $W[1]-difícil$ quando parametrizado pelo tamanho do menor conjunto independente.
\end{teorema}
\begin{proof}
	Observe a seguinte transformação.
	
	O problema da clique multicolorida é conhecidamente $W[1]-difícil$\cite{fellows07}.
	
	Portanto suponha tal $G$ proposto ao problema de clique multicolorida, temos como intenção montar um problema de lista coloração em um grafo $G'$ a partir dele, para tanto seguimos os seguintes passos:
	\begin{itemize}
		\item Para cada cor $i$ presente em $G$ cria-se em $G'$ um vértice $v_i$ (os chamaremos de vértices-cor).
		\item Para cada vértice $u$ em $G$ colorido com a cor $i$, adicionamos à lista do vértice-cor $v_i$ em $G'$ uma cor $c_u$ relacionada a esse vértice (as chamaremos de cores-vértice).
		\item Para cada aresta $e(x,y) \notin E(G)$ onde $x,y \in V(G)$ cria-se em $G'$ um vértice $z_e$ adjacente ao vértice-cor $v_i$ onde $i$ representa as cores de $x$ e $y$, a lista coloração de $z_e$ será formada por $c_x$ e $c_y$.
	\end{itemize}
	É notável que a treewidth de $G'$ é dada por $k$, já que a remoção dos vértices-cor leva a um grafo sem arestas. Assim sendo se $G$ possui uma clique multicolorida podemos facilmente colorir $G'$ da seguinte forma:
	
	Ao vértice-cor $v_i$ atribua a cor-vértice $c_u$ onde $u$ é o vértice colorido com a cor $i$ em $G$. Dessa forma todos os vértices $z_e$ possuem ainda uma cor disponível para sua coloração já que ele representa uma não-aresta em $G$. 
	
	Para a volta observe que uma lista coloração válida em $G'$ implica em uma clique multicolorida em $G$, isso se dá pois dois vértices $x,y$ coloridos com cores diferentes em $G$ não aparecem em uma lista de algum $z_e$ em $G'$ se e somente se existe uma aresta $e(x,y) \in E(G)$, portanto as cores-vértices escolhidas para os vértices $v_i$ são uma respectivamente uma clique formadas por tais $i$ em $G$. Mostramos assim que lista coloração parametrizada por treewidth é $W[1]-difícil$.
	
	Sabemos que coloração em Grafos(2,1) é equivalente a lista coloração em um grafo bipartido, portanto nossa tentativa de parametrizar a coloração de (2,1) pelo tamanho do menor conjunto independente é equivalente a parametrizar lista-coloração em bipartidos pelo tamanho do menor conjunto independente, é de pouca dificuldade ver que a treewidth de um grafo bipartido existe em função do menor independente, mostrando assim que coloração em Grafos(2,1) parametrizada pelo tamanho do menor conjunto independente é $W[1]-difícil$. 
	
\end{proof}

\section{Parametrização pelo tamanho do maior independente}

 Sabemos agora que a parametrização pelo menor independente não nos traz um algoritmo FPT, porém ao analisarmos o comportamento do problema quando parametrizado pelo maior independente vemos que a limitação do tamanho de $r2$ também limita $r1$; Tendo tal limitação a utilização de um método força bruta se mostra uma abordagem válida, como mostrado o seguinte teorema.
\begin{teorema}
  Coloração de Grafos(2,1) é FPT quando parametrizado pelo tamanho do maior conjunto independente.
\end{teorema}
\begin{proof}
  Para tal demonstração onde $k$ é o tamanho de $r2$, observe que são necessárias pelo menos $t$ cores, onde $t$ é a cardinalidade da clique para colorir tal grafo, novamente usaremos a estratégia de transformar coloração de (2,1) em lista coloração de bipartido.
  
  Em uma lista coloração de bipartido, se um vértice possui uma lista com mais cores do que o tamanho de sua vizinhança, ele sempre terá disponível uma cor para sua coloração, podemos portanto remover esse vértice do grafo sem alterar sua coloração, ao chegarmos ao ponto onde todo vértice com tal configuração foi removido temos que $t$ está limitado em função de $k$, portanto rodar um algoritmo de força bruta para encontrar a coloração se mostra FPT.     
\end{proof}

\section{Parametrização pelo tamanho da clique}

Para a demonstração da complexidade parametrizada utilizando $k=\#\ell$ nos voltamos novamente para transformação da clique em um Grafo(2,1) em listas coloração do restante bipartido, dessa forma nosso problema parametrizado original se torna um novo problema, lista coloração de bipartido parametrizado pelo tamanho da paleta de cores. 

Mostraremos no entanto que essa parametrização não é proveitosa já que o problema se mostra equivalente à PreColoring Extension com limite de cores, mostrado ser NP-Completo para grafos bipartidos mesmo quando sua paleta é de tamanho 3\cite{kratochvil94}.

\begin{teorema}
	Lista coloração em bipartidos é NP-Completo
\end{teorema}
\label{theorem:list-coloring-bipartide}
\begin{proof}
	Suponha uma instância $P$ do problema PreColoring Extension e $G$ seu grafo de entrada, sabemos que $G$ possui uma paleta $C$ de cores de tamanho definido, e que existem $v \in V(G)$ que já estão coloridos com uma cor $c \in C$, podemos ver tal configuração como um grafo $G'$ onde os vértices $v$ possuem listas contendo apenas $c$, e os demais vértices possuem listas de tamanho $\#C$ contendo todas as cores, nos levando a um problema de lista coloração $Q$ que tem como entrada $G'$.
	
	Uma coloração possível para $G$ implica em uma coloração possível para $G'$, já que nos basta atribuir aos vértices em $G'$ as mesmas cores atribuídas em $G$. De forma análoga, uma lista coloração possível em $G'$ implica em uma coloração possível em $G$.
\end{proof}


Apesar do tamanho da paleta não ter se mostrado uma escolha adequada, ele levanta novos parametros que são interessantes para o problema de lista coloração em bipartidos, observe pois que, sabemos que Precoloring extension é polinomial se todas as listas tem tamanho 1 ou 2 \cite{hujter93}, e NP-Completo se todas tem listas e tamanho 1 à 3 \cite{kratochvil94}, isso levanta duas formas de se abordar o problema, o que acontece quando o número de vértices com listas de tamanho 1 e 2 varia, e o que acontece quando o número de vertices com listas de tamanho 3 varia.

Mostraremos nas próximas seções como se dão tais comportamentos e como eles se relacionam a coloração de Grafos$(r,\ell)$

\section{Parametrizado pela quantidade de vértices vizinhos à clique}
Nos focaremos nessa seção em grafos(2,1) cuja a clique tenha tamanho 3, já mostrada ser o menor tamanho necessário para que o problema de seja NP-Completo mostrado no teorema \ref{theorem:list-coloring-bipartide} e em \cite{kratochvil94}. Portanto um vértice que é vizinho da clique tem necessariamente uma lista contendo uma ou duas cores, já que um vértice não pertencente a clique que tenha lista de tamanho zero deveria fazer parte da clique, em contrapartida um vértice com lista tamanho 3 é um vértice não vizinho a clique. 

Mostraremos que mesmo quando parametrizado pela quantidade de vértices com listas de tamanho um, dois, ou um e dois o problema é Para-Np-completo. Para tanto é necessário encontrar uma instância do problema já parametrizado cuja solução permanece igualmente difícil.

Portanto essa seção será dividida em três casos, um contendo vértices de listas tamanho um, outro contendo vértices com listas de tamanhos dois, e finalmente contendo listas de tamanho um e dois.
\subsection{Apenas vértices com listas tamanho um}
É importante ressaltar que os seguintes teoremas estabelecem a base para a resolução do problema envolvendo os vizinhos da clique. 
\begin{teorema}
 \label{teorema:6-v-np}
  Seis vértices com lista de tamanho um são necessários para que lista coloração em bipartido seja Para-NP-completo. 
\end{teorema}
\begin{proof}
 Sabemos que em nosso problema temos dois conjuntos independentes, $r1$ e $r2$, também é verdade que exceto pelos citados seis vértices todos os outros vértices tem listas de tamanho três, os vértices de $r1$ podem estar ligados arbitrariamente aos vértices de $r2$. 
 
 Observe a disposição da figura \ref{fig:seis-vertices-lista-um}
 
\begin{figure}[H]
		\centering
		\includesvg[scale=0.6]{seis-vertices-lista-um.svg}
		\caption{Uma possível instância formada por 6 vértices com listas tamanho 1. }
		\label{fig:seis-vertices-lista-um}
\end{figure}

Observe como a presença de um vértice com uma lista de tamanho um influencia em sua vizinhança; a existência desse vértice implica na remoção de sua única possível cor das listas de seus vizinhos, a estrutura mostrada na figura \ref{fig:seis-vertices-lista-um} mantém a coloração do restante NP-Completo ao restringir a cor dos vizinhos sem ter o controle da estrutura resultante perdendo portanto o viés parametrizado.

\end{proof}

Interessantemente o problema é de trivial solução quando o número de vértices com listas tamanho um é zero, já que se torna o problema de 3-coloração em bipartidos, mas NP-Completo com 6 vértices, queremos portanto notar qual é o menor número de vértices no qual o problema é NP-Completo, e consequentemente Para-NP-Completo para nossa parametrização.

\begin{teorema}
\label{lemma:1-col-P}
 Lista coloração em bipartido é de solução trival quando há apenas um vértice de lista tamanho um.
\end{teorema}
\begin{proof}
 Sabemos que além do vértice citado todos os outros vértices têm listas de tamanho três dessa forma basta que o conjunto independente no qual tal vértice está inserido seja colorido com a única cor escolhida para o vértice e o conjunto independente sobrante pode ser colorido com qualquer cor.
\end{proof}

\begin{teorema}
 Lista coloração em bipartido é de solução linear quando existem dois vértices de lista tamanho um.
\end{teorema}
\begin{proof}
 Para essa demonstração é necessária a observação em que existem duas possíveis configurações para essa instância:
 \begin{itemize}
   \item Ambos os vértices pertencem ao mesmo conjunto independente.
   \item Os vértices pertencem a conjuntos distintos.
 \end{itemize} 
 No primeiro caso a estratégia usada no lema \ref{lemma:1-col-P} pode ser adaptada para a solução. Para tanto basta colorir tais vértices com suas cores disponíveis e o conjunto independente ao qual pertencem com a cor de algum deles, e o conjunto sobressalente com a cor restante.
 
 No segundo caso, a coloração também é simples. Se tais vértices tem cores distintas basta colorir seus respectivos conjuntos com a mesma cor. Se não, como temos três cores podemos colorir os vértices com a cor 1, um conjunto com a cor 2 e os demais vértices com a cor 3.  
\end{proof}

\begin{teorema}
  Três vértices com lista de tamanho um são suficientes para que lista coloração em bipartido seja NP-completo.
\end{teorema}
\begin{proof}
  Mostraremos aqui como que três vértices são suficientes para que o problema seja NP-Completo, esse resultado se dá pois é possível reproduzir a estrutura do teorema \ref{teorema:6-v-np} utilizando os ditos 3 vértices, para tanto basta que notar o seguinte \emph{gadget}:
  
\begin{figure}[H]
  \begin{subfigure}
    \centering
		\includesvg[scale=0.6]{gadget-1.svg}
  \end{subfigure}
  \begin{subfigure}
    \centering
		\includesvg[scale=0.6]{gadget-2.svg}
  \end{subfigure}
  \begin{subfigure}
    \centering
		\includesvg[scale=0.6]{gadget-3.svg}
  \end{subfigure}
  \caption{Gadget com vértices de lista um reproduzindo vértice de lista um em vértice de lista três}
\end{figure}

Usando tal gadget usando dois vértices com lista um em um vértice com lista três é possível reproduzir um vértice de cor um, sendo assim tendo três vértices de lista um é possível obter seis vértices de lista um e reproduzir a estratégia mostrada no teorema \ref{teorema:6-v-np}, que nos mostra a NP-Completude desse problema.
  
\end{proof}
 Dado os resultados apresentados nessa seção mostramos portanto que o numéro de vértices vizinhos a dois dos três vértices pertencentes a clique não é um parâmetro viável para uma solução FPT.
  
\subsection{Vértices com listas de tamanho dois}
Mostraremos nessa seção que vértices com listas tamanho dois não são suficientes para que um algoritmo FPT seja extraído.

Como já visto o problema é de trivial solução quando todos os vértices tem listas de tamanho três, portanto precisamos ainda encontrar qual número de vértices de tamanho dois onde o problema se torna NP-Completo.

\begin{teorema}
 Seis vértices com listas tamanho 2 são necessários e suficientes para que lista-coloração em bipartido seja NP-Completo.
\end{teorema}
\begin{proof}
 Para mostrarmos que qualquer número de vértices abaixo de 6 é insuficiente, mostraremos que a menos que existam pelo menos 3 vértices com lista tamanho dois em cada conjunto independente, a coloração é simples de ser feita. 
 
 Se um conjunto independete contém apenas dois vértices com listas tamanho dois, podemos afirmar que todos os vértices nesse conjunto compartilham uma cor em suas listas, podendo colorir tal conjunto com essa cor, todos os outros vértices ainda têm pelo menos uma cor disponível para sua coloração podendo ser colorido com ela.
 
 Para completar nossa demonstração basta encontrar uma configuração onde o problema de lista coloração permanece NP-Completo.
 observaremos agora a vizinhança dos vértices com lista dois, iremos isolar as instâncias em alguns casos, separando sem perda de generalidade um vértice.
 \begin{itemize}
   \item Vizinhança de tamanho um.
   Nesse caso podemos notar que independentemente do vértice e seu vizinho eles sempre compartilharão uma cor, colorimos o vértice de $r2$ com tal cor, além disso como conheçemos a vizinhança sabemos que nenhum outro vértice é vizinho deste, podemos então colorir os restantes vértices de $r2$ com a cor remanescente, dessa forma uma das três cores ainda resta e podemos a usar para colorir o restante do $r1$.
   \begin{figure}[H]
     \centering
     \fontsize{6}{10}
     \includesvg[scale=0.5]{1-edge.svg}
     \caption{Demonstração de coloração para vizinhança de tamanho 1.}
   \end{figure}
 \end{itemize}
\end{proof}

\subsection{Vértices com listas de tamanho um e dois}

Para o acontecimento de haver vértices com listas tamanho um e dois, basta notar que podemos pintar aqueles que contém listas de tamanho um, que propagará a remoção de sua cor aos vizinhos, e ao realizar isso iterativamente, acabaremos com um caso em que todos os vértices terão ou listas de tamanho dois ou três, caindo em algm dos casos supracitados.

\section{Parametrizado pela quantidade de vértices não vizinhos a clique}
Como visto na seção anterior, os vértices que não são vizinhos a clique, quando transformados em vértices do problema de lista coloração se transformam em vértices com listas tamanho três, portanto nosso desejo é resolver lista coloração em bipartidos com listas de tamanho um a três parametrizado pela quantidade de vértices com lista de tamanho três, a solução deriva do seguinte teorema.

\begin{teorema}
 Lista coloração em bipartidos com listas de tamanho um a três é FPT quando parametrizado pela quantidade de vértices com lista de tamanho três
\end{teorema}
\begin{proof}
 Dado que temos $k$ vértices com 3 escolhas cada é possível montar um algoritmo de busca em árvore de altura limitada de tamanho $3^k$, e então executar o algoritmo linear proposto em \cite{hujter93} obtendo um algoritmo $\mathcal{O}(3^kn)$
 
\end{proof}
