\chapter{Conclusão} \label{cap:conclusao}
Apresentamos aqui o desenvolvimento e especificações da dificuldade do problema de coloração em grafos$(r,\ell)$. Ao nos aprofundar na investigação percebemos ainda a inesperada inaptidão da maior parte dos parâmetros relacionados ao tamanho das partições para a extração de um algoritmo FPT. Nas seções seguintes sumarizaremos nossos resultados e mostraremos consequências e questionamentos relacionados.

\section{Resultados e consequências}
Obtivemos no capítulo 2 uma dicotomia $\P/\NPc$ para o problema de coloração em grafos$(r,\ell)$, é conhecido que o problema de coloração em um grafo $G$ pode ser visto como um problema de clique cover em seu complemento $G'$\cite{gareyjohnson}. DAssim sendo podemos estender a dicotomia $\P/\NPc$ da tabela \ref{tab:tabela_dictrl} para o problema de clique cover em grafos$(r,\ell)$ simplesmente trocando as linhas pelas colunas da tabela, dessa forma obtemos a seguinte dicotomia $\P/\NPc$:
\begin{table}[!htb]
	\center
	\begin{tabular}{l|*{7}c}
		\toprule
		\backslashbox{$r$}{$l$} & 0 & 1 & 2 & 3 & 4 & \ldots & n\\
		\midrule
		0 & \P & \P & \P & \P & \NPc & \ldots & \NPc\\
		1 & \P & \P & \NPc & \NPc & \NPc & \ldots & \NPc\\
		2 & \P & \NPc & \NPc & \NPc & \NPc & \ldots & \NPc\\
		3 & \NPc & \NPc & \NPc & \NPc & \NPc & \ldots & \NPc\\
		4 & \NPc & \NPc & \NPc & \NPc & \NPc & \ldots & \NPc\\
		$\vdots$ & $\vdots$ & $\vdots$ & $\vdots$ & $\vdots$ & $\vdots$ & $\ddots$ & \NPc\\
		n & \NPc & \NPc & \NPc & \NPc & \NPc & \ldots & \NPc\\
		\bottomrule
	\end{tabular}%
	\caption{Dicotomia $\P/\NPc$ do problema de clique cover em Grafos$(r,\ell)$}%
\end{table}%

Já no capítulo 3 exploramos com profundidade o comportamento do problema em Grafos$(2,1)$. Obtendo dessa forma 2 algoritmos FPT, uma demonstração de $W[1]-dificuldade$ e algumas de para-NP-completude, tendo colateralmente mostrado resultados clássicos e parametrizados para o problema de lista coloração em bipartidos.

\section{Trabalhos futuros}
Alguns questionamentos levantados durante a produção deste trabalho que tomaram forma de possíveis trabalhos futuros foram:
\begin{itemize}
  \item Existe uma relação entre as parametrizações da coloração de Grafos$(2,1)$ e clique cover de Grafos$(1,2)$?
  \item Quais são as características que afetam parametrizações em Grafos$(r,\ell)$ diferentes de Grafos$(2,1)$?
  \item Existe algum parâmetro que seja possível extrair um algoritmo FPT para qualquer $r$ ou $\ell$?
\end{itemize}
