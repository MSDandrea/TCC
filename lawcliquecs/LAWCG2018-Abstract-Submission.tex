\documentclass[12pt]{article}
%=====================================

\usepackage{amssymb, amsmath, mathtools}
\usepackage{xspace}

\newcommand{\leaves}[1]{\ensuremath{\operatorname{leaves}(#1)}\xspace}

\begin{document}
\pagestyle{empty}
%=====================================

\begin{center}
\Large

{\bf The {Colorability} problem on ($r$,$\ell$)-graphs and a few parametrized solutions }\\[0.2in]

\large
%
%
%IMPORTANT: the asterisk * indicates the author who will present the work.
%
M. S. D. Alves$^{1,*}$\hspace{.2cm}
U. S. Souza$^1$\hspace{.2cm}

%
$^1$ Universidade Federal Fluminense

\end{center}
%=======================================
\normalsize
\vspace{-2.0ex}

%
\noindent\line(1,0){390}\\
%
{\it Keywords: colorability; graph coloring; ($r$,$\ell$)-graph; list coloring; parametrized complexity}\\%; Computational complexity}\\
%
\line(1,0){390}\\
\vspace{-2.0ex}
%==========================================
\sloppy

An \emph{($r$,$\ell$)-graph} is a graph that can be partitioned into $r$ independent sets and $\ell$ cliques; In 
the \textsc{$k$-Colorability} problem we are asked to determine whether a given graph $G$ admits a vertex coloring using at most $k$ colors such that adjacent vertices have different colors.

In this work, we describe a \emph{Poly vs NP-complete} dichotomy of this problem regarding to the parameter $r$ and $\ell$ of ($r,\ell$)-graphs, determining the boundaries of the NP-completenes for such a class. In addition, we analyze the complexity of the problem on ($r,\ell$)-graphs under the parametrized complexity perspective. 
 
A parameterized problem $(\Pi,k)$ is said \emph{fixed-parameter tractable} (FPT) if it can be solved in time $f(k)\times n^{O(1)}$, where $f$ is an arbitrary function, and $n$ is the size of the input. 
 
Using a reduction from \textsc{$k$-Colourability} on ($r$,$\ell$)-graph to \textsc{List-Coloring} as strategy, we are able to discovery that given a (2,1)-partition of the input graph $G$, to finding an optimal coloring of $G$ is: W[1]-hard when parametrized by the size of the smallest independet part; Para-NP-complete when parametrized by the size of the complete part; FPT when parametrized by the number of vertices having no neighbors in the complete part; and FPT when the size of the complete part and the size of the smallest independent part are agregated parameters.
 
%=========================================
\vspace{-2ex}
%\renewcommand\refname{\footnotesize \bf References\vspace{-2ex}}
%\begin{thebibliography}{4}
%\itemsep=-1pt
%\footnotesize
% \bibitem{SBPOUeverton} Dourado C. M., Oliveira A. R., Protti, F., Souza, S. U.: Conex\~{a}o de terminais com n\'{u}mero restrito de roteadores e elos, Annals of XLVI SBPO, pp. 2965--2976, 2014.\vspace{-2ex}%
%\end{thebibliography}
\end{document}
