\documentclass{article}
\usepackage[utf8]{inputenc}
\usepackage{tikz}

\title{Relatório para Projeto Final I}
\author{
    Matheus Souza D'Andrea Alves \\
    \texttt{mad@id.uff.br}
}
\date{Dezembro 2017}

\begin{document}

\maketitle

\section{Título do Projeto}
Coloração em Grafos(r,l)

\section{Participantes}

Aluno: Matheus Souza D'Andrea Alves
\\
Orientador: Dr. Uéverton dos Santos Souza

\section{Objetivo do Projeto}

O trabalho tem como proposta desvendar e catalogar a complexidade parametrizada de tal problema para a classe O trabalho a seguir tem como proposta desvendar e catalogar a complexidade clássica e parametrizada de tal problema para a classe de Grafos(r,l), i.e. grafos particionáveis em r conjuntos independentes e l cliques; Identificando as características que tornam o problema difícil e a relação do problema de coloração com outros problemas, quando abordado pela perspectiva parametrizada.

\section{Resumo do trabalho realizado ao longo do semestre e descrição da situação atual}

Realizamos a especificação do problema de coloração quanto a sua complexidade extendidos para quaisquer r e l, olhando a partir de uma perspectiva de análise clássica, começamos a investigar a complexidade parametrizada para o caso específico dos Grafos(2,1).

\section{Cronograma}

\begin{tabular}{|c|p{11cm}|}
    \hline
    Data de entrega &  Ação \\ \hline
    01/01/2018 & Revisão da literatura \\ \hline
    15/01/2018 & Analisar a complexidade de list 3-coloring em bipartido, quando parametrizado pelo nº de vértices com lista de tamanho 3 \\ \hline
    05/02/2018 & Analisar a complexidade de list 3-coloring em bipartido, quando parametrizado pelo nº de vértices com lista de tamanho até 2 \\ \hline
    13/02/2018 & Definir a fronteira do caso Para-NP-Completo \\ \hline
    05/03/2018 & Analisar a a complexidade de Grafo(2,1) considerando K e S1 como parâmetro \\ \hline
    12/03/2018 & Submeter um artigo ao SBPO com os resultados até então obtidos \\ \hline
    12/03/2018 & Submeter ao Vasconcelos Torres \\ \hline
    --/05/2018 & Analisar a complexidade parametrizada de outras famílias de Grafos(r,l) \\ \hline
    --/05/2018 & Submissão e revisão do material finalizado para apresentação do TCC \\ \hline
\end{tabular}

\section{Nota final atribuída pelo orientador: \underline{\hspace{3cm}}}

\section{Assinaturas: do(s) aluno(s) e do Orientador}


\begin{center}
    \vspace{1cm}
    \underline{\hspace{10cm}} \\
    \texttt{Matheus S. D'Andrea Alves}
    \\
    \vspace{1cm}
    \underline{\hspace{10cm}} \\
    \texttt{Uéverton dos Santos Souza}
    
\end{center}


\end{document}
