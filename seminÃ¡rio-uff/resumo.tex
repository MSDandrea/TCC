\documentclass[a4paper,11pt]{article}
% packages --------------------------
\usepackage[T1]{fontenc}
\usepackage[brazilian]{babel}
\usepackage[utf8]{inputenc}
\usepackage{lmodern}
\usepackage{amsthm}
\usepackage{diagbox} %tabelas com barras
\usepackage{booktabs}
\usepackage{graphicx}			% Inclusão de gráficos
% ------------------------------------
% config
\newtheorem{definição}{Definição}
\newtheorem{teorema}{Teorema}
\newtheorem{corolario}{Corolário}
\graphicspath{{../figuras/}}
\newcommand{\Mod}[1]{\ (\mathrm{mod}\ #1)}

% ------------------------------------

\title{Coloração de grafos(r,l)}
\author{Matheus Souza D'Andrea Alves e Uéverton dos Santos Souza}

\begin{document}
  \maketitle
  A intenção do trabalho é a de explorar e elaborar uma dicotomia para o problema de Coloração mínima em Grafos(r,l) (i.e. grafos que podem ser particionados em r conjuntos independentes e l cliques) quanto a sua complexidade.
  
  Para tanto começaremos os estudos a partir de conhecimentos simples sobre coloração e particionamento de grafos e avançaremos as descobertas demonstrando a intimicidade do problema de coloração mínima em grafos(r,l) e lista coloração em grafos(r,l), demonstraremos particulamente como a afirmação: Se lista coloração é NP-Completo para Grafos(r,l) então coloração mínima é NP-Completo para Grafos(r,l+1), é verdadeira.
  
  Ao final do trabalho mostramos que existe uma dicotomia clara para o problema de coloração mínima na classe dos grafos(r,l) e levantamos perguntas sobre suas características e possibilidade de resolução dos problemas NP-Completos no domínio parametrizado.
\end{document}
