                %!TEX program = xelatex
\documentclass[9pt, compress]{beamer}
\usetheme[sectionpage=progressbar]{metropolis}

\usepackage{booktabs}  
\usepackage[scale=2]{ccicons}
\usepackage[T1]{fontenc}
\usepackage[brazilian]{babel}
\usepackage[utf8]{inputenc}
\usepackage{lmodern}
\usepackage{amsthm}
\usepackage{diagbox} %tabelas com barras
\usepackage{booktabs}
\usepackage{graphicx}			% Inclusão de gráficos

\author{\textbf{Matheus S. D'Andrea Alves}, \textbf{Uéverton dos Santos Souza} } 
\title{Minimum vertex coloring on Graphs(r,l)}
%\subtitle{}
%\logo{}
\institute{\textbf{Universidade Federal Fluminense}}
\date{August 2017}
%\subject{}
%\setbeamercovered{transparent}
%\setbeamertemplate{navigation symbols}{}
\begin{document}
    \maketitle
    \begin{frame}{Overral}
    \centering
        \tableofcontents
    \end{frame}
    \section{Introduction}
    \begin{frame}
        \frametitle{Who am I}
        \begin{columns}
          \begin{column}{0.3\textwidth}
            \includegraphics[scale=0.2]{../figuras/foto.jpg}
          \end{column}
          \begin{column}{0.5\textwidth}
            Undergraduated student at:
            \begin{itemize}
              \item Universidade Federal Fluminense
            \end{itemize}
            
            Interested in:
            \begin{itemize}
              \item Graph theory
              \item Complexity analysis
              \item Software engineering
            \end{itemize}
            
            You can find me here:
            \begin{itemize}
              \item \url{github.com/MSDandrea}
              \item \url{telegram.me/MSDandrea}
            \end{itemize}
          \end{column}
        \end{columns}
    \end{frame}
    \section{The problem}
    \begin{frame}{Basic concepts}
      \begin{columns}
        \begin{column}{0.5\textwidth}
          \textbf{Graph(r,l)}
          
          Any graph in the class of graphs that can be partitionated in $r$ cliques and $l$ independent sets
        \end{column}
        \begin{column}{0.5\textwidth}
          \textbf{Minimum vertex coloring}
          
          A minimum vertex coloring is an assignment of a color among $k$ colors to each vertex of a graph such that no edge connects two identically colored vertices and $k$ is the smallest value to obtain a k-coloring
        \end{column}
      \end{columns}
    \end{frame}
    \section{Our approach}
    \begin{frame}
        \frametitle{Who am I}
        \begin{columns}
          \begin{column}{0.5\textwidth}
            este é apenas um teste 
          \end{column}
          \begin{column}{0.5\textwidth}
            este é apenas um teste 2891473894712893749712398479182379481729384
          \end{column}
        \end{columns}
    \end{frame}
    \section{First results}
    \begin{frame}
        \frametitle{Who am I}
        \begin{columns}
          \begin{column}{0.5\textwidth}
            este é apenas um teste 
          \end{column}
          \begin{column}{0.5\textwidth}
            este é apenas um teste 2891473894712893749712398479182379481729384
          \end{column}
        \end{columns}
    \end{frame}
    \section{The relation between list coloring and minimum coloring in Graph(r,l)}
    \begin{frame}
        \frametitle{Who am I}
        \begin{columns}
          \begin{column}{0.5\textwidth}
            este é apenas um teste 
          \end{column}
          \begin{column}{0.5\textwidth}
            este é apenas um teste 2891473894712893749712398479182379481729384
          \end{column}
        \end{columns}
    \end{frame}
    \section{Final results}
    \begin{frame}
        \frametitle{Who am I}
        \begin{columns}
          \begin{column}{0.5\textwidth}
            este é apenas um teste 
          \end{column}
          \begin{column}{0.5\textwidth}
            este é apenas um teste 2891473894712893749712398479182379481729384
          \end{column}
        \end{columns}
    \end{frame}
    \section{Conclusion}
     \subsection{Future works}
\end{document}
